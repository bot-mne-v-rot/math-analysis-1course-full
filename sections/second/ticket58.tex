% !TEX root = ../MatanColloc02.tex

\subsection{Теорема Больцано–Коши. Теорема о непрерывном образе отрезка \href{https://youtu.be/E7inz4tp-6k?t=8060}{\Walley}}
\begin{theorem-non}
    Больцано-Коши (Если функция непрерывна на отрезке $[a, b]$, то она принимает все значения между $f(a)$ и $f(b)$)

    $f: [a, b] \longrightarrow \R$ непрерывна на $[a, b], C$ лежит между $f(a)$ и $f(b)$

    Тогда $\exists c\in (a, b)$, т.ч. $f(c) = C$
    \begin{proof} От противного. $C \notin f([a, b])$.

        $[a, b] \subset \R$ -- связное мн-во. Т.к. непрерывный образ связного 
        мн-ва -- связное мн-во, $f([a, b])$ -- связное. 
    
        $U := (-\infty, C)$, $V := (C, +\infty)$ -- открытые мн-ва.
        $U \cap V = \varnothing$. НУО, $f(a) < f(b) \Rightarrow
        f(a) \in U, f(b) \in V$, т.к. $C$ лежит между $f(a)$ и $f(b)$
        $\Rightarrow U \neq \varnothing, V \neq \varnothing$.
    
        $f([a, b]) \subset U \cup V \Rightarrow$ либо $f([a, b]) \subset U$,
        либо $f([a, b]) \subset V$ $\Rightarrow$ либо $f(b) \notin f([a, b])$,
        либо $f(a) \notin f([a, b])$. Противоречие.
    
    \end{proof}
\end{theorem-non}

\begin{theorem-non}
    Непрерывный образ отрезка - отрезок 

    \begin{proof} \quad

        $f:[a, b] \longrightarrow \R$.
        %TODO: Делать уоминания теорем ссылочными на место, где мы их определеяем
        Тогда по теореме Вейерштрасса она достигает своих точных верхней и нижней граней.
        \begin{itemize}
            \item[] $m:= \min\limits_{x \in [a, b]}{f(x)}$ \qquad тогда $m = f(u)$
            \item[] $M:= \max\limits_{x \in [a, b]}{f(x)}$ \qquad тогда $M = f(v)$
        \end{itemize} для некоторых $u, v \in [a, b]$

        Посмотрим на $f$ на отрезке $[u, v]$. Тогда по теореме Больцано-Коши она принимает все значения между $f(u) = m$
        и $f(v) = M$. Тогда $[m, M] \subset f([a, b]) \subset [m, M] \Longrightarrow f([a, b]) = [m, M]$ 
    \end{proof}
\end{theorem-non}