% !TEX root = ../MatanColloc02.tex

\subsection{Связность. Непрерывный образ связного множества. Связность отрезка \href{https://youtu.be/E7inz4tp-6k?t=6544}{\Walley}}


\begin{conj}
    $(X, \rho)$ -- метрическое пространство.
    
    $X$ -- \textbf{несвязное}, если существуют непустые открытые $U, V$, 
    т.ч. $U \cap V = \varnothing$ и $X = U \cup V$.
\end{conj}

\notice $X$ -- несвязное, если существует $U$ -- непустое
открытое, т.ч. $X \setminus U$ -- непустое открытое.

\begin{conj}
    $X$ -- \textbf{связное} $\Longleftrightarrow$ если $X = U \cup V$, 
    $U, V$ -- открытые и $U \cap V = \varnothing$, то $U$ или $V$ пустое.
\end{conj}

\begin{conj}
    $A \subset X$ -- \textbf{связное}, если $(A, \rho)$ -- связное.
\end{conj}

\begin{theorem-non}
    Непрерывный образ связного множества -- связное множество.
\end{theorem-non}
\begin{proof} $ $

    $f: X \rightarrow Y$ непрерывно, и $X$ связно. Пусть $f(X) \subset
    U \cup V$, $U, V$ -- открытые и $U \cap V = \varnothing \Rightarrow
    X \subset f^{-1}(U) \cup f^{-1}(V)$, $f^{-1}(U)$, $f^{-1}(V)$ --
    открытые, т.к. $f$ непрерывно, $f^{-1}(U) \cap f^{-1}(V) =
    \varnothing \Rightarrow X$ -- целиком лежит в одном из них.
    НУО, $X \subset f^{-1}(U) \Rightarrow f(X) \subset U$.
\end{proof}

\begin{theorem-non}
$[a, b] \subset \R$ -- связное множество.
\end{theorem-non}
т.е. если $[a, b] \subset U \cup V$, $U \cap V = \varnothing$, $U, V$
открыты, то либо $[a, b] \subset U$, либо $[a, b] \subset V$.

\begin{proof} $ $

    Пусть $b \in V$. Посмотрим теперь на $[a, b] \cap U =: S$. Если 
    $S$ пустое, значит весь отрезок $[a, b]$ накрывается мн-вом $V$,
    что подходит под условие теоремы. Пусть $S \neq \varnothing$.

    Пусть $y := \sup S$. $b$ -- верхняя граница $S$ $\Rightarrow$
    $y \leqslant b \Rightarrow y \in [a, b]$.

    Рассмотрим несколько случаев:
    \begin{enumerate}
        \item $\mathbf{y \in V}$:
        
        $V$ открыто $\Rightarrow (y - \varepsilon,\,\, y + \varepsilon)
        \subset V$ для некоторого $\varepsilon > 0$. $V \cap U =
        \varnothing \Rightarrow (y - \varepsilon,\,\, y + \varepsilon)
        \cap U = \varnothing \Rightarrow (y - \varepsilon,\,\, y + \varepsilon)
        \cap S = \varnothing \Rightarrow y - \varepsilon$ -- верхняя 
        граница для $S$, но $y = \sup S$. Противоречие.

        \item $\mathbf{y \in U}$:
        
        $y \in U, b \in V, y \leqslant b \Rightarrow y < b$. $U$ -- открытое 
        $\Rightarrow (y - \varepsilon,\,\, y + \varepsilon) \subset U$ для 
        некоторого $\varepsilon > 0$. Можно сузить этот интервал так,
        чтобы $\varepsilon < b$.
        
        Тогда $\begin{cases}
            [y, y + \varepsilon) \subset [a, b] \\
            [y, y + \varepsilon) \subset U
        \end{cases} \Rightarrow [y, y + \varepsilon) \subset S$,
        но $y = \sup S$. Противоречие.
    \end{enumerate}

    Таким образом, $S = \varnothing$ и $[a, b] \subset V$.
\end{proof}