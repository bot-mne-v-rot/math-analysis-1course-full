\subsection{Неравенства между синусом и аргументом. Непрерывность тригонометрических функций. Предел $\lim{\frac{sin x}{x}}$. Непрерывность обратных тригонометрических функций \href{https://youtu.be/an3AiCY2hPE?t=1960}{\Walley}}
\begin{theorem-non}
    При $0 < x < \frac{\pi}{2} \qquad \sin{x} < x < \tan{x}$
    \begin{proof} \quad 
        
        \begin{tikzpicture}[scale=3.0,cap=round,>=latex]
            % draw the coordinates
            \draw[->] (-1.5cm,0cm) -- (1.5cm,0cm) node[right,fill=white] {$x$};
            \draw[->] (0cm,-1.5cm) -- (0cm,1.5cm) node[above,fill=white] {$y$};
    
            % draw the unit circle
            \draw[thick] (0cm,0cm) circle(1cm);

            %draw angle
            \draw[thick,orange] ([shift=(10:1cm)]-0.8,-0.171) arc (0:45:0.18cm);

            %draw lines
            \draw[blue] (0cm,0cm) -- (45:1.41cm)
                 (45:1cm) -- (0:0.71cm)
                 (45:1.41cm) -- (0:1cm)
                 (0cm,0cm) -- (1cm,0cm);

            %draw dots
            \filldraw[black] (45:1cm) circle(0.4pt)
                     (45:1.41cm) circle(0.4pt)
                     (0cm:0cm) circle(0.4pt)
                     (0:1cm) circle(0.4pt)
                     (0:0.71cm) circle(0.4pt);

            %draw nodes
            \draw (-0.1cm,-0.1cm) node(z) {$0$}
                  (0.71cm,-0.1cm) node(h) {$H$}
                  (1.1cm,-0.1cm) node(b) {$B$}
                  (0.55cm,0.7cm) node(a) {$A$}
                  (0.85cm,1cm) node(c) {$C$}
                  (0.25cm,0.1cm) node(x) {$x$};
            
            \foreach \x/\xtext in {
                180/\pi}
                    \draw (\x:0.85cm) node[fill=white] {$\xtext$};

            \draw (-1.25cm,0cm) node[above=1pt] {$(-1,0)$}
                  (1.25cm,0cm)  node[above=1pt] {$(1,0)$}
                  (0cm,-1.25cm) node[fill=white] {$(0,-1)$}
                  (0cm,1.25cm)  node[fill=white] {$(0,1)$};
        \end{tikzpicture}

        $\sin{x} = AH \\
        \tg{x} = BC \\ 
        S_{\triangle AOB} = \frac{1}{2} OB \cdot AH = \frac{\sin{x}}{2}$

        $S_{\triangle BOC} = \frac{1}{2} OB \cdot CB = \frac{\tg{x}}{2}$
        
        $S_{\text{сектор } AOB} = \pi \cdot \frac{x}{2\pi} = \frac{x}{2}$

        $S_{\triangle AOB} = \frac{\sin{x}}{2} < S_{\text{сектор } AOB} = \frac{x}{2}< S_{\triangle BOC} = \frac{\tg{x}}{2}$
    \end{proof} 
\end{theorem-non}
    \follow \begin{enumerate}
        \item $\abs{\sin{x}} \leqslant \abs{x} \qquad x \in \R$ и если $x \neq 0$, то неравенство строгое
        \begin{proof}
            $x > 0$ Если $x < \frac{\pi}{2}$, то $\sin{x} < x$. Если $x \geqslant \frac{\pi}{2}$, то $\sin{x} \leqslant 1 < \frac{\pi}{2} \leqslant x$

            $\abs{\sin{(-x)}} = \abs{-\sin{x}} = \abs{sin{x}} \leqslant \abs{x} = \abs{-x}$
        \end{proof}
        \item $\abs{\sin{x} - \sin{y}} \leqslant \abs{x - y}$ и $\abs{\cos{x} - \cos{y}} \leqslant \abs{x - y} \quad \forall x, y \in \R$
        \begin{proof}
            $\abs{\sin{x} - \sin{y}} = 2\abs{\sin{\frac{x-y}{2}}} \cdot \abs{\cos{\frac{x+y}{2}}} \leqslant 2\abs{\sin{\frac{x-y}{2}}} \leqslant 2\abs{\frac{x-y}{2}} = \abs{x-y}$
        \end{proof}
    \end{enumerate}

    \begin{theorem-non} \quad 

        \begin{enumerate}
            \item $sin$ и $cos$ равномерно непрервны на $\R$
            \begin{proof}
                $\abs{\sin{x} - \sin{y}} \leqslant \abs{x - y}$

                Если $\abs{x - y} < \varepsilon$, то $\abs{\sin{x} - \sin{y}} < \varepsilon$
            \end{proof}
            \item $tg$ и $ctg$ непрерывны на своей области определения
            \begin{proof}
                $tg = \frac{sin}{cos}$. Если $tg$ определен, то есть его знаменатель не равен 0, то он непрерывен, так как 
                отношение непрерывных функций непрерывно, с $ctg$ аналогично  
            \end{proof}
        \end{enumerate}
    \end{theorem-non}

    \begin{theorem-non}
        Первый замечательный предел. $\lim\limits_{x \rightarrow 0}{\frac{\sin{x}}{x}} = 1$

        \begin{proof}
            $\sin{x} < x < \tg{x} = \frac{\sin{x}}{\cos{x}}$ при $0 < x < \frac{\pi}{2}
            \Longrightarrow \cos{x} < \frac{\sin{x}}{x} < 1$ при $x \in \left(-\frac{\pi}{2}, \frac{\pi}{2} \right)$
            $x \neq 0$. Устремим $x$ к 0, тогда $\cos{x} \longrightarrow 1 \Longrightarrow 1 < \frac{\sin{x}}{x} < 1 \Longrightarrow \frac{\sin{x}}{x} \longrightarrow 1$
        \end{proof}
    \end{theorem-non}

    \subsection*{Обратные тригонометрические функции}
    \begin{itemize}
        \item[] $sin: [-\frac{\pi}{2}, \frac{\pi}{2}] \longrightarrow [-1, 1]$ \quad непрерывен, строго возрастает
        \item[] $cos: [0, \pi] \longrightarrow [-1, 1]$ \quad непрерывен, строго убывает
        \item[] \qquad $arcsin: [-1, 1] \longrightarrow [-\frac{\pi}{2}, \frac{\pi}{2}]$ \quad непрерывен, строго возрастает 
        \item[] \qquad $arccos: [-1, 1] \longrightarrow [0, \pi]$ \quad непрерывен, строго убывает 
        \item[] $tg: \left(-\frac{\pi}{2}, \frac{\pi}{2}\right) \longrightarrow \R$ \quad непрерывен, строго возрастает 
        \item[] $ctg: (0, \pi) \longrightarrow \R$ \quad непрерывен, строго убывает
        \item[] \qquad $arctg: \R \longrightarrow \left(-\frac{\pi}{2}, \frac{\pi}{2}\right)$ \quad непрерывен, строго возрастает
        \item[] \qquad $arcctg: \R \longrightarrow (0, \pi)$ \quad непрерывен, строго убывает 
    \end{itemize}