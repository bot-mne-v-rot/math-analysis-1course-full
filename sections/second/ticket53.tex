% !TEX root = ../MatanColloc02.tex

\subsection{Непрерывность функций, действующих в $\R^d$. Характеристика непрерывности в терминах прообразов \href{https://youtu.be/E7inz4tp-6k?t=1769}{\Walley}}


\begin{theorem-non}
\end{theorem-non}
$f : E \rightarrow \mathbb{R}^d$, $a \in E$. \\
Тогда $f$ непрерывна в точке $a$ $\Longleftrightarrow$ все координатные
функции непрерывны в точке $a$.

\begin{proof} Содержательный случай: $a$ -- предельная точка.

    $f$ непрерывна в точке $a$ $\Longleftrightarrow$ $\lim f(x_n) = f(a)$
    для любой последовательности точек $x_n \rightarrow a$
    $\Longleftrightarrow$ $f_k(x_n) = f_k(a) \,\, \forall k = 1..d$
    $\Longleftrightarrow f_k$ непрерывна в $a$

\end{proof}

\begin{theorem-non}
\end{theorem-non}

$f : X \rightarrow Y$ $X, Y$ -- топологические пространства.
$f$ непрерывна во всех точках $X$ $\Longleftrightarrow$
$\forall U \subset Y$ открытое $f^{-1}(U)$ открытое множество.

\begin{proof} $ $

    \textbf{``$\Longrightarrow$'':}

    Возьмём $U$ -- открытое, и $a \in f^{-1}(U)$.
    Докажем, что $a$ -- внутренняя точка $f^{-1}(U)$.
    
    $a \in f^{-1}(U) \Rightarrow f(a) \in U \Rightarrow \exists
    \varepsilon > 0 : B_{\varepsilon}(f(a)) \in U \Rightarrow
    \exists \delta > 0 : f(B_{\delta}(a)) \subset B_{\varepsilon}(f(a))
    \subset U \Rightarrow f(B_{\delta}(a)) \subset U \Rightarrow
    B_{\delta}(a) \subset f^{-1}(U) \Rightarrow$ $a$ -- внутренняя точка
    $f^{-1}(U)$.

    \textbf{``$\Longleftarrow$'':}

    Возьмём $U = B_\varepsilon (f(a))$, оно открыто $\Rightarrow$
    $f^{-1}(U)$ -- открытое.

    $a \in f^{-1}(U) \Rightarrow a$ -- внутренняя точка $\Rightarrow
    \exists \delta > 0 : B_{\delta}(a) \subset f^{-1}(U) \Rightarrow
    f(B_{\delta}(a)) \subset U = B_\varepsilon(f(a)) \Rightarrow f$
    непрерывна в точке $a$.
\end{proof}