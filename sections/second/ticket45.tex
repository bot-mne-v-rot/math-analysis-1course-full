\subsection{Лемма Лебега. Связь между компактностью и секвенциальной компактностью \href{https://youtu.be/IwHCWoW4oes?t=4113}{\Walley}} 

\begin{lemma}
    Лемма Лебега \\
    
    Пусть $K$ - секвенциально компактное множество
    и $K \subset \bigcup \mathcal{U_{\alpha}}$, где $\mathcal{U_{\alpha}}$ --- открытые мн-ва. \\
    Тогда существует такое $\varepsilon > 0$, что $\forall x \in K$ шарик $B_{\varepsilon}(x)$ целиком покрывается каким-то $\mathcal{U_{\alpha}}$.
    \begin{proof}
        От противного. Пусть $\varepsilon = \frac{1}{n}$ не подходит. \\
         $B_{\frac{1}{n}}(x_n)$ не содержится целиком ни в одном из $\mathcal{U_{\alpha}}$.  \\
         Поскольку у нас секвенциальная компактность, выберем из $x_n$ сходящуюся подпоследовательность $x_{n_k} \rightarrow y \in K$.
         Тогда $y \in \mathcal{U}_{{\alpha}_0}$ --- открытое множество, тогда $\exists \; \varepsilon > 0,$ т.ч. $B_{\varepsilon}(y) \subset \mathcal{U}_{{\alpha}_0}$. \\
         Так как $\rho(x_{n_k},y)\rightarrow 0$, начиная с некоторого номера $\rho(x_{n_k},y) < \frac{\varepsilon}{2}$ и $\frac{1}{n_k} < \frac{\varepsilon}{2}$, \\
         тогда $B_{\frac{1}{n_k}}(x_{n_k}) \stackrel{?}{\subset} B_{\varepsilon}(y) \subset \mathcal{U}_{{\alpha}_0}$. \\
         $B_{\frac{1}{n_k}}(x_{n_k}) \subset B_{\frac{\varepsilon}{2}}(x_{n_k}) \stackrel{?}{\subset} B_{\varepsilon}(y)$ \\
         Возьмём $z \in B_{\frac{\varepsilon}{2}(x_{n_k})} \Longrightarrow 
         \begin{cases}
             \rho(x_{n_k},y) < \frac{\varepsilon}{2} \\
             \rho(z,x_{n_k}) < \frac{\varepsilon}{2}
         \end{cases} \Longrightarrow$ 
        $\rho(z,y) \leqslant \rho (z, x_{n_k}) + \rho (x_{n_k},y) < \varepsilon \Longrightarrow z \in B_{\varepsilon}(y)$ \\
        Противоречие.
    \end{proof}

\end{lemma}

\begin{theorem-non}
    В метрическом пространстве компактность $=$ секвенциальная компактность.

    \begin{proof}
        "$\Longleftarrow:$" Возьмем покрытие $K$ открытыми множествами $\mathcal{U_{\alpha}}, \quad K \subset \bigcup \mathcal{U_{\alpha}}$ \\
        Возьмём $\varepsilon > 0$ из Леммы Лебега. Рассмотрим покрытие $K \subset \bigcup\limits_{x \in K} B_{\varepsilon}(x)$. \\
        Достаточно выбрать конечное подпокрытие из $\bigcup\limits_{x \in K}B_{\varepsilon}(x)$ \\
        Возьмём любой $x_1 \in K$, если $B_{\varepsilon}(x_1) \supset K$, то нашлось конечное подпокрытие. \\
        Иначе выберем $x_2 \in K \setminus B_{\varepsilon}(x_1)$, если $B_{\varepsilon}(x_1) \cup B_{\varepsilon}(x_2) \supset K$, то нашлось конечное подпокрытие \\
        Иначе $x_3 \in K \setminus (B_{\varepsilon}(x_1) \cup B_{\varepsilon}(x_2))$ и т.д.  \\
        Если в какой то момент процедура оборвалась, то нашлось конечное подпокрытие. \\
        Если он бесконечный, то $x_n \in K \setminus \bigcup\limits_{j = 1}^{n - 1} B_{\varepsilon}(x_j) \Longrightarrow x_n \notin B_{\varepsilon}(x_j)$ при $j < n$ 
        $\Longrightarrow \rho(x_n, x_j) \geqslant \varepsilon \quad \forall n > j \Longrightarrow$ пусть $x_{n_k}$ --- сходящаяся подпоследовательность, значит $x_{n_k}$ --- фундаментальна, а это не так
        $\Longrightarrow$ у неё нет сходящайся подпоследовательности. Это противоречит секвенциальной компактности.
    \end{proof}

\end{theorem-non}