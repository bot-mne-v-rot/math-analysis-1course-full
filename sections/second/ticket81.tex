\subsection{Иррациональность числа $e$ \href{https://youtu.be/au9-34CerJM?t=7867}{\Walley}}

\begin{theorem-non}
    $e$ --- иррационально    

    \begin{proof}
        $2 < e < 3 \Longrightarrow e$ --- не является целым.
        
        Пусть $e = \frac{m}{n}, \quad$ где $n \geqslant 2$ (т.к. $e$ --- не целое)
        
        Формула Тейлора с остатком в форме Лагранжа для $x_0 = 0, x = 1:$
        \begin{center}
            $e^1 = 1 + 1 + \frac{1}{2!} + \frac{1}{3!} + \dots + \frac{1}{n!} + \frac{e^c}{(n+1)!}$ где $c \in (0,1)$ \\
            $\underbrace{\frac{m}{n} \cdot n!}_{\text{целое}} = n! \cdot e = 
            \underbrace{n! + n! + \frac{n!}{2!} + \frac{n!}{3!} + \dots + \frac{n!}{n!}}_{\text{целое}} + \frac{e^c}{n+1}$

            $\Longrightarrow \frac{e^c}{n+1} \in \Z \Longrightarrow \frac{e^c}{n+1} \geqslant 1$ 

            $\frac{e^c}{n + 1} \leqslant \frac{e}{3} < 1$ (т.к. $c \in (0,1), n \geqslant 2$) 
        \end{center}
        Противоречие $\Longrightarrow e$ --- иррационально. 

    \end{proof}

\end{theorem-non}