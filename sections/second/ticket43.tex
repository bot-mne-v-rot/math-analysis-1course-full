\subsection{Теорема о пересечении семейства компактов. Следствие о вложенных компактах \href{https://youtu.be/IwHCWoW4oes?t=2250}{\Walley}}

\begin{theorem-non}
    Теорема о пересечении семейства компактов.

    $K_{\alpha}$ - семейство компактов, таких что пересечение любого их конечного количества непусто.
    Тогда $\bigcap K_{\alpha} \neq \varnothing$ --- пересечение всех непусто.
    
    \begin{proof}
        Возьмём компакт $K_{\alpha_0}$. Предположим, что $\bigcap K_{\alpha} = \varnothing$. \\
        Тогда $K_{\alpha_0} \subset (X \setminus \bigcap\limits_{\alpha \neq \alpha_{0}} K_{\alpha}) = 
        \bigcup\limits_{\alpha \neq \alpha_{0}} \underbrace{(X \setminus K_{\alpha})}_\text{откр. мн-ва}$ --- покрытие $K_{\alpha_{0}}$ открытыми множествами \\
        $\Longrightarrow$ из него можно выделить конечное подпокрытие: $X \setminus K_{\alpha_{1}}, \dots, X \setminus K_{\alpha_{n}}$ \\
        $\Longrightarrow K_{\alpha_0} \subset \bigcup\limits_{j=1}^{n}(X \setminus K_{\alpha_{j}}) = X \setminus \bigcap\limits_{j = 1}^{n}(K_{\alpha_{j}}) \Longrightarrow
        \bigcap\limits_{j=0}^{n} K_{\alpha_{j}} = \varnothing$ противоречие. 
    \end{proof}

\end{theorem-non}

\follow \quad Если $K_1 \supset K_2 \supset K_3 \supset ...$ непустые компакты, то $\bigcap\limits_{n=1}^{\infty} K_n \neq \varnothing$ \\
Если у нас компакты вложены, то пересечение их конечного количества --- самый маленький из них, по условию он не пустой, значит пересечение непусто.