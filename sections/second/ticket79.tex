% !TEX root = ../MatanColloc02.tex

\subsection{Формулы Тейлора с остатком в форме Лагранжа. Следствия \href{https://youtu.be/au9-34CerJM?t=5698}{\Walley}}

\begin{theorem-non}
    $f: \langle a, b \rangle \rightarrow \R$

    $n + 1$ раз дифференцируема на $(a, b), \; x_0, x \in 
    \langle a, b \rangle$

    Тогда существует $c$, лежащая между $x$ и $x_0$, т.ч.
    $f(x) = T_{n,x_o} f(x) + \frac{f^{(n+1)}(c)}{(n+1)!} \cdot (x - x_0)^{n+1}$

    ($c$ подбирается для фиксированных $x$ и $x_0$)

    \begin{proof}
        Зафиксируем $x$ и возьмём $M$, т.ч.
        
        \begin{gather*}
            f(x) = T_{n,x_0} f(x) + M \cdot (x - x_0)^{n+1}
        \end{gather*}

        Введём вспомогательную функцию:
        \begin{gather*}
            g(t) := f(t) - T_{n, x_0}f(t) - M \cdot (t - x_0)^{n+1} \\
            g^{(n+1)}(t) = f^{(n+1)}(t) - M \cdot (n + 1)!
        \end{gather*}
        
        Надо доказать, что $g^{(n+1)}(c) = 0$ для некоторого $c \in (x_0, x)$ (НУО $x_0 < x$)

        \begin{gather*}
            g^{(k)}(x_0) = 0, \quad k = 0,1, \dots, n \\
            g(x) = 0
        \end{gather*}
        
        По т. Ролля$: \; \exists \; c_1 \in (x_0, x): \; g'(c_1) = 0$ \\
        По т. Ролля для $g': \; \exists \; c_2 \in (x_0, c_1): \; g''(c_2) = 0$

        $\dots$

        По т. Ролля для $g^{(n)}: \; \exists \; c_{n+1} \in (x_0, c_n): \; g^{(n+1)}(c_{n+1}) = 0$

    \end{proof}

\end{theorem-non}

\begin{follow}
    \begin{enumerate}
        \item Если $|f^{(n+1)}(t)| \leqslant M \; \forall t \in (x, x_0)$, то $|R_{n, x_0}f(x)| \leqslant \frac{M(x-x_0)^{n + 1}}{|n+1|)!} = O((x - x_0)^{n + 1})$
            \begin{proof}
                $| \frac{f^{(n+1)}(c)}{(n+1)!} \cdot (x - x_0)^{n+1} | \leqslant M \cdot \frac{|x-x_0|^{n+1}}{(n+1)!}$
            \end{proof}
        \item
        Если \; $|f^{(n)}(t)| \leqslant M \; \forall n \in \N, \forall t \in <a, b>$, то \; $T_{n,x_0}f(x) \rightarrow f(x), \; n \rightarrow \infty$ \\

        $\sum_{k=0}^{n} \frac{f^{(k)}(x_0)}{k!} (x - x_0)^k \rightarrow f(x),$ т.е. $f(x) = \sum_{k=0}^{\infty} \frac{f^{(k)}(x_0)}{k!} \cdot (x - x_0)^k$
            \begin{proof}

                \begin{gather*}
                    |f(x) - T_{n,x_0}f(x)| = |R_{n, x_0} f(x)| \leqslant \frac{M|x-x_0|^{n+1}}{(n+1)!} \stackrel{?}{\rightarrow} 0 \\
                \end{gather*}

                Это так, потому что у нас $x$ и $x_0$ фиксированные, то это просто какое-то число, возведённое в степень, которое делится на факториал, а мы
                уже доказывали, что $\frac{k^n}{n!} \rightarrow 0$

            \end{proof}
    \end{enumerate}
\end{follow}