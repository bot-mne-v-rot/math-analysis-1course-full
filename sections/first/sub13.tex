\subsection{Предел числовой последовательности и предел последовательности в метрическом пространтстве}

    \begin{conj}
        Предел числовой последовательности \\
        $x_1, x_2, x_3 ... \in R$. $a=\lim x_n$ если вне любого интервала, содержащего $a$, содержится лишь конечное число членов последовательности.
    \end{conj}
    \notice -
    Можно рассматривать симметричные интервалы (если есть несимметричный, для удобства его можно расширить или сузить до симметричного)
    
    \begin{conj}
        Предел последовательности в метрическом пространстве
        $(X, d)$ - метрическое пространство, $x_1, x_2 ... \in X$. $a=\lim x_n$ если вне любого шара $B_\varepsilon (a)$ содержится лишь конечное число членов последовательности.
    \end{conj}
    
    \notice -
    Верно также для любого открытого множества, содержащего $a$
    
    \notice -
    Cуществование предела зависит от пространства (в $R_+ x_n = 1/n$ не имеет предела)
    
    \begin{theorem-non}
        Свойства:
        \begin{enumerate}
            \item Если $a=\lim x_n$ и из $x_n$ выкинули какое-то число членок так, чтобы осталось бесконечное число членов, то у оставшейся последовательности тот же предел
            \item Если $a=\lim x_n$ и к последовательности добавить конечное число членов, то $a$ - все еще предел
            \item Добавление, замена или выкидывание конечного количества членом не меняет предел и его наличие (то же самое другими словами)
            \item Перестановка членов не влияет на предел последовательности
            \item Если $a=\lim x_n$ и $a=\lim y_n$, то если их перемешать, то у новой последовательности тоже предел $a$
            \item Если $a=\lim x_n$, тогда у последовательности, в которой $x_n$ встречается с конечной кратностью, тот же предел (написать один и тот же элемент много раз подряд)
        \end{enumerate}
    \end{theorem-non}
    \begin{conj}
        $a=\lim x_n$, если
    \[ \forall \varepsilon>0 \exists N : \forall n\geqslant N d(x_n, a)<\varepsilon\]
    \end{conj}
    \begin{conj}
        $A \subset X, (X, d)$ - метрическое пространство
        $A$ - ограничено, если $A$ целиком содержится в каком-нибудь шаре
    \end{conj}
    
    \begin{theorem-non}
        \quad \\
        \begin{enumerate}
            \item Предел единственный
            \begin{proof}
                \quad \\
                Пусть $a\neq b \Longrightarrow \exists B_{r_1}(a), B_{r_2}(b) : B_{r_1} \cap B_{r, 2} = \varnothing$. \\
                Вне $B_{r_1}(a)$ конечное число членов \\
                Вне $B_{r_2}(b)$ конечное число членов \\
                Тогда в последовательности конечное число членов. Противоречие.
            \end{proof}
            \item Если последовательность имеет предел, то она ограничена
            \begin{proof}
                \quad \\
                Возьмем $\varepsilon=1$. Тогда $\exists N: \forall n\geqslant N\ x_n \in B_1(a)$. \\
                Тогда $r:= \max\{d(a, x_1), d(a, x_2),..., d(a, x_N)\}+1$
            \end{proof}
            \item $a=\lim x_n \Longleftrightarrow \lim d(x_n, a)=0$
            \begin{proof}
                \quad \\
                $\lim d(x_n, a)=0 \Longleftrightarrow \forall \varepsilon > 0 \exists N : \forall n \geqslant N d(x_n, a) < \varepsilon \Longleftrightarrow \lim x_n = a$
            \end{proof}
            \item Если $a=\lim x_n$ и $b = \lim y_n$, то $\lim d(x_n, y_n) = d(a, b)$
            \begin{proof}
                \quad \\
                $d(a, b) \leqslant d(a, x_n)+d(x_n, y_n)+d(y_n, b)$
                $d(x_n, y_n) \leqslant d(a, x_n)+d(a, b)+d(b, y_n) \Longrightarrow$
                $|d(x_n, y_n)-d(a, b)|\leqslant d(x_n, a)+d(y_n, b)$ \\
                Справа каждая меньше $\varepsilon /2$, тогда слева стремится к нулю
            \end{proof}
        \end{enumerate}
    \end{theorem-non}