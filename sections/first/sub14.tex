\subsection{Связь между пределами и предельными точками}
    
    \begin{theorem-non}
        $a$ - предельная точка $A \Longleftrightarrow$ найдется последовательность точек $x\neq a \in A : \lim x_n = a$. Супер очевидно из соответствующих определений, но распишу
        \begin{proof}
            \quad \\
            ``$\Longleftarrow$'': \quad Пусть $x_n \in A$ и $\lim x_n = a$. \\
            Тогда в $B_r(a)\setminus \{a\}$ содержится бесконечное количество точек из $x_n$, так как $\exists N : \forall n \geqslant N x_n \in B_r(a)$ \\
            ``$\Longrightarrow$'': \quad
            $r_1 = 1 \Longrightarrow \exists x_1 \in B_1(a), r_2 = \min\{1/2, d(a, x_1)\}, r_3=\min\{1/3, d(a, x_2)\}...$ \\
            $\forall \varepsilon>0\ \exists N: 1/N<\varepsilon \Longrightarrow \forall n\geqslant N\ d(x_n, a) < 1/n \leqslant 1/N < \varepsilon$
        \end{proof}
    \end{theorem-non}
    
    
    
    \begin{theorem-non}
        Если $x_n \in A$ и $a=\lim x_n$, то $a \in Cl(A)$
    \end{theorem-non}
    \begin{proof}
        Либо $a\in A$, тогда $a \in Cl(A)$, иначе $x_n \neq a$, тогда по теореме 1. $a \in A' \Longrightarrow a \in Cl(A)$
    \end{proof}