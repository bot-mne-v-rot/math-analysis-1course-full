\subsection{Простейшие свойства сходящихся рядов.}

\begin{enumerate}
    \item Сумма ряда единственна
    
    \begin{proof}
        Утверждение про единственность предела частичных сумм
    \end{proof}

    \item Расстановка скобок не меняет суммы ряда (если она была)
    
    \begin{proof}
        $\underbracket[0pt][2pt]{x_1}_{S_1} + (x_2 + x_3 +
        \underbracket[0pt][2pt]{x_4}_{S_4}) + (x_5 + 
        \underbracket[0pt][2pt]{x_6}_{S_6}) + (x_7 + x_8 + 
        \underbracket[0pt][2pt]{x_9}_{S_9})...$

        Т.е. из последовательности частичных сумм просто выбрали
        другую подпоследовательность, ну таким образом, если предел был,
        то он такой же и остался.
    \end{proof}

    \textbf{Замечание.} Он расстановки скобок сумма ряда могла
    появиться.

    Пример. Ряд $1 - 1 + 1 - 1 + 1 - 1 + 1 - 1 + \dots$ расходится.
    Но при расстановке следующим образом скобок:
    $(1 - 1) + (1 - 1) + (1 - 1) + (1 - 1) + \dots$ получаем, что ряд
    имеет сумму $0$.

    \item Добавление/отбрасывание конечного числа членов не влияет на
    сходимость, но влияет на сумму.

    \begin{proof}
        Рассмотрим отбрасывание.

        Ряд $x_1 + x_2 + x_3 + \dots$, частичная сумма которого
        $S_n$, переделали в $x_{k+1} + x_{k+2} + x_{k+3} + \dots$,
        частичная сумма которого $\widetilde{S}_n := x_{k+1} + x_{k+2}
        + \dots + x_{k+n} = S_{k + n} - S_{k}$. Т.к. $k$ фиксировано
        отсюда видно, что если $S_n$ (не) имеет предел, то и
        $\widetilde{S}_n$ (не) имеет предел, и наоборот.

        Добавление - просто обратная операция.
    \end{proof}

    \item Если $\sum_{n = 1}^{+\infty} a_n$ и $\sum_{n = 1}^{+\infty} b_n$
    сходятся, то $\sum_{n = 1}^{+\infty} (a_n \pm b_n)$ сходится и
    $\sum_{n = 1}^{+\infty} (a_n \pm b_n) =\vspace*{0,2cm} \\ = \sum_{n = 1}^{+\infty} a_n
    \pm \sum_{n = 1}^{+\infty} b_n$

    \item Если $\sum_{n = 1}^{+\infty} a_n$ сходится, то
    $\sum_{n = 1}^{+\infty} c a_n$ сходится и $\sum_{n = 1}^{+\infty} 
    c a_n = c \cdot \sum_{n = 1}^{+\infty} a_n$ 
\end{enumerate}