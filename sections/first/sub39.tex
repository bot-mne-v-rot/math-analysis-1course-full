\subsection{Характеристика верхних и нижних пределов с помощью 
    $N$ и $\varepsilon$. Сохранение неравенств.}
    
    \begin{theorem-non}\end{theorem-non}
    \begin{enumerate}
        \item $a = \underline{\lim}\, x_n \in \mathbb{R} \Leftrightarrow
        \begin{cases}
            \forall \varepsilon > 0 \,\, \exists N : \forall n \geqslant N
            \,\, x_n > a - \varepsilon\\
            \forall \varepsilon > 0 \,\, \forall N \,\, \exists n \geqslant N
            : x_n < a + \varepsilon\\
        \end{cases}$
        \item $b = \overline{\lim}\, x_n \in \mathbb{R} \Leftrightarrow
        \begin{cases}
            \forall \varepsilon > 0 \,\, \exists N : \forall n \geqslant N
            \,\, x_n < b + \varepsilon \quad \circled{1}\\
            \forall \varepsilon > 0 \,\, \forall N \,\, \exists n \geqslant N
            : x_n > b - \varepsilon \quad \circled{2}\\
        \end{cases}$
    \end{enumerate}
    \begin{proof} $ $
    
        2. Докажем \circled{1} $\Leftrightarrow \forall \varepsilon > 0\,\, 
        \exists N : z_N < b + \varepsilon$
    
        ''$\Longrightarrow$'':
        \[\forall \varepsilon > 0 \,\, \exists N : \forall n \geqslant N
        \,\, x_n < b + \varepsilon \Rightarrow
        \forall \varepsilon > 0 \,\, \exists N : \forall n \geqslant N
        \,\, x_n < b + \frac{\varepsilon}{2} \Rightarrow\] \[\Rightarrow
        z_N = \sup \{x_N, x_{N + 1}, \dots\} \leqslant b + \frac{\varepsilon}{2}
        < b + \varepsilon \Rightarrow \forall \varepsilon > 0\,\, 
        \exists N : z_N < b + \varepsilon\]
    
        ''$\Longleftarrow$'':
        \[\text{Зафиксируем } \varepsilon > 0
        \Rightarrow \exists N : z_N < b + \varepsilon \Leftrightarrow
        \sup\{x_N, x_{N + 1}, \dots\} < b + \varepsilon \Rightarrow
        x_n < b + \varepsilon \,\, \forall n \geqslant N\]
    
        Докажем \circled{2} $\Leftrightarrow \forall \varepsilon > 0 \,\,
        \forall N \,\, z_N > b - \varepsilon$
    
        ''$\Longrightarrow$'':
        \[\forall \varepsilon > 0 \,\, \forall N \,\, \exists n \geqslant N
        : x_n > b - \varepsilon \text{ при этом } z_N = \sup\{
        x_N, x_{N+1}, x_{N+2}, \dots\} \Rightarrow \forall
        \varepsilon > 0 \,\, \forall N \,\, z_N > b - \varepsilon\]
    
        ''$\Longleftarrow$'':
        \[\text{Зафиксируем } \varepsilon > 0 \text{ и } N \Rightarrow
        z_N > b - \varepsilon \Leftrightarrow \sup\{ x_N, x_{N+1}, \dots\}
        > b - \varepsilon \Rightarrow \exists n \geqslant N : x_n > b -
        \varepsilon,\] \[\text{иначе } \forall n \geqslant N : x_n \leqslant b -
        \varepsilon \text{ и тогда } \sup\{ x_N, x_{N+1}, \dots\} \leq
        b - \varepsilon \Leftrightarrow z_N \leqslant b - \varepsilon\]
    
        \circled{1} + \circled{2} $\Leftrightarrow
        \begin{cases}
            \forall \varepsilon > 0\,\, 
            \exists N : z_N < b + \varepsilon\\
            \forall \varepsilon > 0 \,\,
            \forall N \,\, z_N > b - \varepsilon\\
        \end{cases}$
        $\Leftrightarrow$ т.к. $z_n \searrow$
        $\begin{cases}
            \forall \varepsilon > 0\,\, 
            \exists N : \forall n \geqslant N \,\, z_n < b + \varepsilon\\
            \forall \varepsilon > 0 \,\,
            \forall N \,\, z_N > b - \varepsilon\\
        \end{cases}$
    
        Это и есть определение предела $\Rightarrow b = 
        \overline{\lim}\, x_n$
    
        В обратную сторону, первая строка следует из определения предела,
        вторая строка следует из того, что $(z_n) \searrow$. Более того,
        $(z_n) \searrow$, $\lim z_n = b \Rightarrow z_n \geqslant b$
    
    \end{proof}
    
    \begin{theorem-non}\end{theorem-non}
    Если $x_n \leqslant y_n$, то 
    $\underline{\lim}\, x_n \leqslant \underline{\lim}\, y_n$ и  
    $\overline{\lim}\, x_n \leqslant \overline{\lim}\, y_n$
    
    \begin{proof} $ $
    
    $x_n \leqslant y_n \Rightarrow \inf\{x_n, x_{n + 1}, ...\} \leq
    \inf\{y_n, y_{n + 1}, ...\} \Rightarrow$ по пред. переходу 
    $\underline{\lim}\, x_n \leqslant \underline{\lim}\, y_n$
    
    Аналогично для $\overline{\lim}\, x_n \leqslant \overline{\lim}\, y_n$.
    \end{proof}