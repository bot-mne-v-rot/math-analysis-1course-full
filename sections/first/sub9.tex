\subsection{Внутренние точки. Внутренность множества}
\begin{conj}
    $(X, \beta)$ - метрическое пространство $A \subset X$ \\
    $a \in A, \; a$ - \underline{внутренняя точка множества}, если $B_r(a) \subset A$ для некоторого $r > 0$
    (Открытое множество - такое множество, у которого все точки внутренние) \\
    \underline{Внутренность множества} - множество всех его внутренних точек. Обозначается как $Int A$ \\
\end{conj}
\begin{theorem-non}
    Свойства внутренности: 
    \begin{enumerate}
        \item $Int A \subset A$
        \item $Int A = \bigcup \{G: G \subset A $ и $G$ - открытое$\} =: B$
        \begin{proof}
            \quad \\
            $\bullet \quad Int A \supset B$ \\
            Возьмем $b \in B$. Тогда найдется открытое $G_{\circ} \subset A$, такое, что $b \in G_{\circ} \Longrightarrow$\\
            $\exists r_{>0}$, такой, что $B_r(b) \subset G_{\circ} \subset A \Longrightarrow b$ - внутренняя точка $A$ \\
            $\bullet \quad Int A \subset B$ \\
            Возьмем $a \in Int A \Longrightarrow a$ - внутренняя точка $\Longrightarrow $ открытое множество $ B_r(a) \subset A$ для некоторого $r_{>0}
            \Longrightarrow a \in B_r(a) \subset A$ \\
            $a \in B_r(a) \subset B \Longrightarrow a \in B$
        \end{proof}
        \item $Int A$ - самое большое (по включению) открытое множество, содержащееся в $A$
        \item $Int A$ - открытое множество
        \item $Int A = A \Longleftrightarrow A$ - открытое 
        \item $A \subset B \Longrightarrow Int A \subset Int B$
        \begin{proof}
            Пусть $a \in Int A \Longrightarrow B_r(a) \subset A$ для 
            некоторого $r_{>0} \Longrightarrow a$ - внутренняя точка $B$
        \end{proof}
        \item $Int(A \cap B) = Int A \cap Int B$
        \begin{proof}
            \quad \\
            ``$\subset$'' : $A \cap B \subset A \Longrightarrow Int(A \cap B) \subset Int A$. Это следует из предыдущего пункта. Аналогично для $B$\\ 
            ``$\supset$'' : Пусть $c \in Int A \cap Int B \Longrightarrow 
            \begin{cases}
                c$ - внутренняя точка $A \\
                c$ - внутренняя точка $B 
            \end{cases} \Longrightarrow 
            \begin{cases}
                B_{r_1}(c) \subset A \\
                B_{r_2}(c) \subset B
            \end{cases}$ \vspace{0,2cm} \\ для некоторых $r_1, r_2 > 0 \Longrightarrow 
            B_r(c) \subset A \cap B,$ где $r = min\{r_1, \; r_2\} \Longrightarrow \\ c$ - внутренняя точка $A \cap B$
        \end{proof}
        \item $Int(Int A) = Int A$
        \begin{proof}
            $Int A$ - открытое множество, а внутренность открытого множества совпадает с ним
        \end{proof}
    \end{enumerate}
\end{theorem-non}