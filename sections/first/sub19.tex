\subsection{Векторное пространство. Пространство $R^d$. Скалярное произведение. Неравенство Коши-Буняковского}
    
\begin{conj}
    $X$ - векторное пространство (над полем $\mathbb{R}$), если:
    Определена операции "+": $X\times X \to X$
    "*": $\mathbb{R}\times X \to X$ \\
    \begin{enumerate}
        \item Сложение коммутативно и ассоциативно
        \item Существует $\overrightarrow{0}$
        \item Существует обратный элемент $x+(-x)=\overrightarrow{0}$
        \item $(\alpha \beta)x = \alpha(\beta x)\ \forall \alpha, \beta \in \mathbb{R}\ \ \forall x\in X$
        \item $(\alpha+\beta)x = \alpha x + \beta x$
        \item $\alpha(x+y) = \alpha x + \alpha y$
    \end{enumerate}
\end{conj}
\vspace*{0,5cm}

\begin{conj}
    \quad \\
    $R^d = \{ \langle x_1, x_2,...,x_d\rangle  \}: x_i \in \mathbb{R}$

    $\langle x_1,...,x_d\rangle +\langle y_1,...,y_d\rangle =\langle x_1+y_1,..., x_d + y_d\rangle $

    $\alpha \langle x_1,...,x_d\rangle  = \langle \alpha x_1,..., \alpha x_d\rangle $
\end{conj}

\begin{conj}
    Скалярное произведение $\langle \bullet, \bullet\rangle X\times X \to \mathbb{R}$

    \begin{enumerate}
        \item $\langle x, x\rangle \geqslant 0,\ \langle x, x\rangle =0 \Longleftrightarrow x=\overrightarrow{0}$
        \item $\langle x, y\rangle = \langle y, x\rangle$
        \item $\langle x+y, z\rangle = \langle x, z\rangle + \langle y, z \rangle $
        \item $\langle \alpha x, y\rangle = \alpha \langle x, y \rangle $
    \end{enumerate}
\end{conj}

\begin{conj}
    Неравенство Коши-Буняковского: $\langle x, y \rangle^2 \leqslant \langle x, x \rangle \langle y, y \rangle$
\end{conj} 

\begin{proof}

$f(t):=\langle x+ty, x+ty\rangle = \langle x, x+ty \rangle + \langle ty, x+ty \rangle =\langle x, x \rangle + t\langle x, y \rangle + t\langle y, x \rangle + t^2\langle y, y \rangle = \langle x, x \rangle + 2t\langle x, y \rangle + t^2\langle y, y \rangle \geqslant 0$. Это всегда неотрицательно, тогда дискриминант неположителен.

\[4t^2\langle x, y \rangle^2 - 4t^2\langle x, x \rangle \langle y, y \rangle \leqslant 0 \Longrightarrow \langle x, y \rangle^2 \leqslant \langle x, x \rangle \langle y, y \rangle\]
\end{proof}