\subsection{Теорема Больцано-Вейерштрасса в $\mathbb{R}$}
    \begin{theorem-non}
        Из любой ограниченной последовательности можно выделить
        сходящуюся подпоследовательность.
    \end{theorem-non}
    \begin{proof}
    ${x_n}$ ограничено $\Rightarrow x_n \in [a; b]$
    
    В каком-то из отрезков $[a; \frac{a + b}{2}]$ и $[\frac{a + b}{2}; b]$
    содержится бесконечное число членов послед.\\
    Назовём этот отрезок $[a_1; b_1]$.
    
    В каком-то из отрезков $[a_1; \frac{a_1 + b_1}{2}]$ и 
    $[\frac{a_1 + b_1}{2}; b_1]$
    содержится бесконечное число членов послед.\\
    Назовём этот отрезок $[a_2; b_2]$.
    
    В каком-то из отрезков $[a_2; \frac{a_2 + b_2}{2}]$ и 
    $[\frac{a_2 + b_2}{2}; b_2]$
    содержится бесконечное число членов послед.\\
    Назовём этот отрезок $[a_3; b_3]$.
    \[...\]
    \[[a; b] \supset [a_1; b_1] \supset [a_2; b_2] \supset
    [a_3; b_3] \supset ...\]
    \[b_n - a_n = \frac{b - a}{2^n} \rightarrow 0\]
    
    Тогда по теореме о стягивающихся отрезках $\lim a_n = \lim b_n = c$
    
    Выберем подпоследовательность. Берём $[a_1; b_1]$, в нём есть
    какой-то член последовательности, назовём его $x_{n_1}$.
    
    В $[a_2; b_2]$ содержится бесконечное число членов последовательности
    $\Rightarrow$ есть член последовательности с номером, большим $n_1$.
    Обозначим его $x_{n_2}$, тогда $n_2 > n_1$.
    \[...\]
    $x_{n_k} \in [a_k; b_k], n_1 < n_2 < n_3 < ...$, значит построили
    подпоследовательность.
    
    \[a_k \rightarrow c, \,\, b_k \rightarrow c \quad a_k \leqslant x_{n_k} \leqslant b_k
    \xRightarrow[]{\text{2 мил.}} \lim x_{n_k} = c \]
    \end{proof}