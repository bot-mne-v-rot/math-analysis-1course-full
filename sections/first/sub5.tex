\subsection{Супремум и инфимум}
\begin{conj}
    \quad \\
    \begin{itemize}
        \item[] $x$ - верхняя граница множества $A$, если $\forall a \in A: a \leqslant x$
        \item[] $y$ - нижняя граница множества $A$, если $\forall a \in A: y \leqslant a$ 
        \item[] Множество ограничено снизу, если существует какая-нибудь нижняя граница
        \item[] Множество ограничено сверху, если существует какая-нибудь верхняя граница
    \end{itemize}
\end{conj}
\begin{conj}
    \quad \\
    Пусть $A$ - ограниченное сверху множество, тогда $sup A$ - наименьшая из его верхних границ
\end{conj}
\begin{conj}
    \quad \\
    Пусть $A$ - ограниченное снизу множество, тогда $inf A$ - наибольшая из его нижних границ
\end{conj}
\begin{theorem-non}
    \quad \\
    \begin{enumerate}
        \item Если $A \subset \R, A \neq \varnothing $ и $ A $ ограничено снизу, то существует единственный $inf A$
        \item Если $A \subset \R, A \neq \varnothing $ и $ A $ ограничено сверху, то существует единственный $sup A$
    \end{enumerate}
\end{theorem-non}
\begin{proof}
    \quad \\
    Докажем (2) \\
    Пусть $B$ - множество всех верхних границ множества $A$, т.е. $\forall a \in A, b \in B: a \leqslant b$ \\
    Тогда по аксиоме полноты всегда найдется такой $c: a \leqslant c \leqslant b$ \\
    $c - sup A$ по определению \\
    Докажем, что $c$ - единсвтенный \\
    Пусть $\exists c_1, c_2 - sup A$ \\
    Тогда если $c_1 < c_2$, то $c_2 \neq sup A$ \\
    Если $c_1 > c_2$,то $c_1 \neq sup A$ \\
    Следовательно, $c_1 = c_2 = sup A \Longrightarrow sup A$ - единсвтенный  
\end{proof}
\follow
\begin{enumerate}
    \item $B \subset A, B \neq \varnothing $ и $ A $ ограничено снизу. Тогда $inf B \geqslant inf A$
    \item $B \subset A, B \neq \varnothing $ и $ A $ ограничено сверху. Тогда $sup B \leqslant sup A$
\end{enumerate}
\begin{proof}
    \quad \\
    Докажем (1) \\
    Пусть $a = inf A$. Тогда $a$ - нижняя граница $A \Longrightarrow \forall x \in
    A : a \leqslant x \Longrightarrow \forall x \in B : a \leqslant x \Longrightarrow \\
    a$ - нижняя граница $B \Longrightarrow a \leqslant inf B$  
\end{proof}
\notice - Теорема неверна без аксиомы полноты \\
\begin{itemize}
    \item[] $A =\{x \in \Q : x^2 < 2\} \Longrightarrow$ в множестве рациональных чисел у $A$ нет супремума
\end{itemize}
\begin{theorem-non}
    \quad \\
    \begin{enumerate}
        \item $a = inf A \Longleftrightarrow 
        \begin{cases}
            a \leqslant x \quad \forall x \in A \\
            \forall \varepsilon > 0 \quad \exists x \in A : x < a + \varepsilon
        \end{cases}$ 
        \item $b = sup A \Longleftrightarrow 
        \begin{cases}
            b \geqslant x \quad \forall x \in A \\
            \forall \varepsilon > 0 \quad \exists x \in A : x > b - \varepsilon
        \end{cases}$ 
    \end{enumerate}
\end{theorem-non}
\notice
\begin{itemize}
    \item Если $A$ неограничено сверху, то $sup A = +\infty$
    \item Если $A$ неограничено снизу, то $inf A = -\infty$
\end{itemize}