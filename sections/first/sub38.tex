\subsection{Верхний и нижний пределы. Связь между частичными пределами и  
    верхним и нижним пределами.}
    
    \begin{conj}
    Нижний и верхний пределы
    \end{conj}
    $x_n$ - числовая последовательность.
    
    $\underline{\lim} x_n := \liminf x_n := \lim \inf_{k \geqslant n} x_k$ -- 
    нижний предел.
    
    $\overline{\lim} x_n := \limsup x_n := \lim \sup_{k \geqslant n} x_k$ -- 
    верхний предел.
    
    $y_n := \inf_{k \geqslant n} x_k = \inf\{x_n, x_{n+1}, x_{n+2}, \dots\}$
    $\quad y_n \leqslant y_{n+1}$\\
    $z_n := \sup_{k \geqslant n} x_k = \sup\{x_n, x_{n+1}, x_{n+2}, \dots\}$
    $\quad z_n \geqslant z_{n+1}$
    
    \begin{theorem-non}
        $\underline{\lim}$ и $\overline{\lim}$ существуют в 
        $\overline{\mathbb{R}}$ и $\underline{\lim} \leqslant \overline{\lim}$
    \end{theorem-non}
    \begin{proof} $ $
    
    Про $\underline{\lim}$: $y_n \leqslant y_{n+1} \Rightarrow (y_n)$ --
    возрастающая последовательность $\Rightarrow$ у неё есть предел в
    $\overline{\mathbb{R}}$.
    
    Про $\overline{\lim}$: $z_n \geqslant z_{n+1} \Rightarrow (z_n)$ --
    убывающая последовательность $\Rightarrow$ у неё есть предел в
    $\overline{\mathbb{R}}$.
    
    Про неравенство $\underline{\lim} \leqslant \overline{\lim}$:
    $y_n \leqslant z_n$, $y_n \rightarrow \underline{\lim}$,
    $z_n \rightarrow \overline{\lim} \Rightarrow$ по предельному
    переходу в неравенстве $\underline{\lim} \leqslant \overline{\lim}$.
    \end{proof}
    
    \begin{theorem-non}\end{theorem-non}
    \begin{enumerate}
        \item $\overline{\lim}$ -- наибольший частичный предел
        \item $\underline{\lim}$ -- наименьший частичный предел
        \item $\exists \lim \in \overline{\mathbb{R}} \Leftrightarrow
        \overline{\lim} = \underline{\lim}$ и в этом случае 
        $\lim = \overline{\lim} = \underline{\lim}$ 
    \end{enumerate}
    \begin{proof} $ $
    
        \begin{enumerate}
            \item $a := \overline{\lim} \, x_n$
            
            Рассмотрим \textbf{случай} $a \in \mathbb{R}$
    
            Докажем, что $a$ -- частичный предел.
    
            $a = \lim z_n$, $z_n = \sup_{k \geqslant n} x_k$, $z_n \searrow a$
    
            Будем строить некоторую подпоследовательность $(x_{n_k})$.\\
            Найдётся $n_k > n_{k-1} : x_{n_k} > a - \frac{1}{k}$.
            Пусть не нашлось $\Rightarrow x_n \leqslant a - \frac{1}{k} \forall
            n \geqslant n_{k-1} \Rightarrow \sup \{x_{n_{k-1}}, x_{n_{k-1} + 1},
            \dots \} \leqslant a - \frac{1}{k} \Rightarrow a \leqslant z_{n_{k - 1}} 
            \leqslant a - \frac{1}{k}$. Противоречие
    
            $a - \frac{1}{k} \rightarrow a$, $z_{n_k} \rightarrow a$,
            $a - \frac{1}{k} < x_{n_k} \leqslant z_{n_k} \xRightarrow[]
            {\text{2 мил.}} x_{n_k} \rightarrow a$
    
            Докажем, что $a$ -- наибольший частичный предел.
    
            Пусть $b$ - частичный предел $\Rightarrow b = \lim x_{n_k}$.
            Но $x_{n_k} \rightarrow b$, $z_{n_k} \rightarrow a \Rightarrow$
            по предельному переходу $b \leqslant a$. 
    
            Рассмотрим \textbf{случай} $a = -\infty$.
    
            Тогда $z_n \rightarrow -\infty$, но $z_n = 
            \sup\{x_n, x_{n+1},\dots\} \geqslant x_n \Rightarrow x_n
            \rightarrow -\infty$.
    
            Рассмотрим \textbf{случай} $a = +\infty$.
    
            Тогда $z_n = +\infty \Rightarrow \sup {x_1, x_2, \dots} =
            +\infty \Rightarrow x_n$ не ограничена сверху $\Rightarrow$
            в ней найдётся подпоследовательность, стремящаяся к $+\infty$.
    
            \item Доказывается аналогично
            
            \item 
            ''$\Longrightarrow$'':
            
            Если $a = \lim x_n$, то все подпоследовательности стремятся
            к $a \Rightarrow$ все частичные пределы равны $a \Rightarrow
            \overline{\lim} x_n = \underline{\lim} x_n = \lim x_n = a$.
    
            ''$\Longleftarrow$'':
    
            $y_n \rightarrow a$, $z_n \rightarrow a$, $y_n \leqslant x_n \leqslant z_n$
            $\xRightarrow[]{\text{2 мил.}} x_n \rightarrow a \Rightarrow
            \lim x_n = \overline{\lim}\, x_n = \underline{\lim}\, x_n = a$
        \end{enumerate}
    \end{proof}
    
    \textbf{Замечание.} Арифметики для верхних и нижних пределов нет.
    
    Пример.
    \[x_n = (-1)^n, \quad y_n = (-1)^{n + 1} \Rightarrow
    \underline{\lim}\, x_n = \underline{\lim}\, y_n = -1\]
    \[x_n + y_n = 0 \Rightarrow \underline{\lim}\, (x_n + y_n) = 
    \underline{\lim}\, (x_n + y_n) = 0\]
    \[\underline{\lim}\, x_n + \underline{\lim}\, y_n = -2 < 0 =
    \underline{\lim}\, (x_n + y_n)\]