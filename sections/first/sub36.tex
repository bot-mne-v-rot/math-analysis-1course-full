\subsection{Фундаментальные последовательности. Критерий Коши.}
    
    \begin{conj}
    Фундаментальная последовательность (сходящаяся в себе,
    последовательность Коши)
    \end{conj}
    Пусть $(X, \rho)$ -- метрическое пространство. $x_n \in X$.
    $x_n$ -- фундаментальная последовательность, если $\forall
    \varepsilon > 0 \,\, \exists N : \forall n, m \geqslant N \,\,
    \rho(x_n, x_m) < \varepsilon$
    
    \textbf{Свойства:}
    \begin{enumerate}
        \item Сходящаяся последовательность фундаментальна.
        
        \textbf{Доказательство:}\\
        Пусть $\lim x_n := a$. Зафиксируем $\varepsilon > 0$.
        Тогда $\exists N :\\ \forall n \geqslant N \,\,\, \rho(x_n, a) < 
        \frac{1}{2} \varepsilon$\\ 
        $\forall m \geqslant N \,\,\, \rho(x_m, a) < \frac{1}{2} \varepsilon$\\
        $\Rightarrow \rho(x_n, x_m) \leqslant \rho(x_n, a) + \rho(x_m, a) <
        \varepsilon$
    
        \item Фундаментальная последовательность ограничена
        
        \textbf{Доказательство:}\\
        Берём $\varepsilon = 1$. Тогда $\exists N : \forall n, m \geqslant N \,\,
        \rho(x_n, x_m) < 1 \Rightarrow$\\
        $\Rightarrow \forall n \geqslant N \,\, \rho(x_n, x_N) < 1
        \Leftrightarrow x_n \in B_1(x_N)$\\
        $R := \max\{\rho(x_1, x_N), \rho(x_2, x_N), \dots, 
        \rho(x_{N-1}, x_N)\} + 1 \Rightarrow \forall n \,\, x_n \in 
        B_R(x_N)$
    
        \item Если у фундаментальной последовательности есть сходящаяся
        подпоследовательность, то фундаментальная последовательность
        имеет тот же предел.
    
        \textbf{Доказательство:}\\
        Пусть $\lim x_{n_k} = a$. Зафиксируем $\varepsilon > 0$.\\
        $\exists K : \forall k \geqslant K \quad \rho(x_{n_k}, a) <
        \frac{1}{2} \varepsilon$\\
        $\exists N : \forall n, m \geqslant N \quad \rho(x_n, x_m) <
        \frac{1}{2} \varepsilon$\\
        Возьмём $N \geqslant 0$ и подберём такое $k$, что $k \geqslant N$\\
        и $n_k \geqslant N$ (например, $k \geqslant \max\{N, K\}$ подходит)\\
        Тогда $\rho(x_n, x_{n_k}) < \frac{1}{2} \varepsilon$
        (т.к. $n_k \geqslant N$)\\
        И тогда $\rho(x_{n_k}, a) < \frac{1}{2} \varepsilon$
        (т.к. $k \geqslant K$)\\
        $\Rightarrow \rho(x_n, a) \leqslant \rho(x_n, x_{n_k}) +
        \rho(x_{n_k}, a) < \varepsilon \Rightarrow \lim x_n = a$
    \end{enumerate}
    
    \begin{theorem-non}Критерий Коши\end{theorem-non}
    Числовая последовательность имеет предел $\Leftrightarrow$
    она фундаментальна.
    
    \begin{proof} $ $
    
    ''$\Longrightarrow$'':\\
    По свойству 1.
    
    ''$\Longleftarrow$'':\\
    фундаментальность $\xRightarrow[]{\text{св-во 2}}$
    ограниченность $\xRightarrow[]
    {\text{Больцано–Вейерштрасса}}$\\
    $\begin{rcases*}
        \Rightarrow \text{сущ. сходящаяся подпосл.}\\
        \quad\quad\text{фундаментальность}
    \end{rcases*}$
    $\xRightarrow[]{\text{св-во 3}}$ существует конечный предел.
    
    \end{proof}