\subsection{Подпоследовательности.
    Теорема о стягивающихся отрезках}
    
    \begin{conj}
    Последовательность $(x_n)$, $n_1 < n_2 < n_3 < ...$ Тогда
    $(x_{n_k})$ - подпоследовательность.
    \end{conj}
    \textbf{Замечание.} $n_k \geqslant k$ (по индукции)
    
    \textbf{Свойства:}
    \begin{enumerate}
        \item Если последовательность имеет предел, то подпоследовательность
        имеет тот же предел.
        \item Пусть две подпоследовательности в объединении дают исходную
        последовательность. Если подпоследовательности имеют одинаковый
        предел, то исходная последовательность имеет тот же предел.
    \end{enumerate}
    
    \begin{theorem-non}О стягивающихся отрезках.\end{theorem-non}
    \[\text{Пусть }[a_1; b_1] \supset [a_2; b_2] \supset [a_3; b_3] 
    \supset ... \text{ и } \lim (b_n - a_n) = 0\]
    Тогда существует единственная точка $c$, принадлежащая всем отрезкам
    и $\lim a_n = \lim b_n = c$.
    \[\text{Т.е. } \bigcap_{n = 1}^{+\infty} [a_n; b_n] = \{c\}\]
    
    \begin{proof}
        По теореме о вложенных отрезках $\bigcap_{n = 1}^{+\infty} [a_n; b_n]
        \neq \varnothing$.
        \[\text{Пусть } c,d \in \bigcap_{n = 1}^{+\infty} [a_n; b_n]
        \Rightarrow c, d \in [a_n; b_n] \forall n; \text{ НУО, } d \geqslant c\]
        \[0 \leqslant d - c \leqslant b_n - a_n \rightarrow 0 \Rightarrow c = d
        \text{, иначе } \exists n : b_n - a_n < \varepsilon = d - c\]
        \[0 \leqslant c - a_n \leqslant b_n - a_n \rightarrow 0
        \xRightarrow[]{\text{2 мил.}}
        c - a_n \rightarrow 0 \Rightarrow \lim a_n = c\]
        \[0 \leqslant b_n - c \leqslant b_n - a_n \rightarrow 0
        \xRightarrow[]{\text{2 мил.}}
        b_n - c \rightarrow 0 \Rightarrow \lim b_n = c\]
    \end{proof}