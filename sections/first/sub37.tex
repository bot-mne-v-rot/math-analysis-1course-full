\subsection{Теорема Больцано–Вейерштрасса в $\mathbb{R}^d$.
    Полнота $\mathbb{R}^d$ }
    
    \begin{conj}
    Полнота метрического простраства
    \end{conj}
    Пусть $(X, \rho)$ -- метрическое пространство.
    $X$ - полное, если любая фундаментальная последовательность
    в нём имеет предел.
    
    \begin{theorem-non}
        $\mathbb{R}^d$ - полное пространство.
    \end{theorem-non}
    \begin{proof} $ $
    
        Возьмём фундаментальную последовательность $(x_n)$.
        $x_n = (x_n^{(1)}, x_n^{(2)}, \dots, x_n^{(d)})$
    
        \[\forall \varepsilon > 0 \,\, \exists N :
        \forall n, m \geqslant N \,\, \rho(x_n, x_m) <
        \varepsilon \Rightarrow\]
        \[\Rightarrow \abs{x_n^{(k)} -
        x_m^{(k)}} \leqslant \sqrt{(x_n^{(1)} -
        x_m^{(1)})^2 + (x_n^{(2)} - x_m^{(2)})^2 +
        \dots + (x_n^{(d)} - x_m^{(d)})^2} < \varepsilon
        \Rightarrow\] \[ \Rightarrow
        \text{числовая послед. } x_n^{(k)}
        \text{ фундаментальна } \Rightarrow \text
        {у неё есть конечный предел}\] \[\lim x_n^{(k)}
        = a_k \Rightarrow \lim x_n = a, \quad a = 
        (a_1, a_2, \dots, a_d) \]
        Т.к. в $\mathbb{R}^d$ покоординатная и
        сходимость по метрике -- одно и то же.
    
    \end{proof}
    
    \begin{theorem-non}
    Больцано–Вейерштрасса в $\mathbb{R}^d$.
    \end{theorem-non}
    \begin{proof}
    Пусть векторная последовательность
    $x_n = (x_n^{(1)}, x_n^{(2)}, \dots, x_n^{(d)})$ ограничена.
    Это равносильно тому, что все её координатные последовательности
    ограничены.
    
    Выделим из первой координатной последовательности сходящуюся
    подпоследовательность $(x_{n_{1, k}}^{(1)})$. Тогда получим
    подпоследовательность $(x_{n_{1, k}})$, первая координатная
    последовательность которой сходится, а остальные ограничены.
    
    Тогда в ней можно выделить такую подпоследовательность
    $(x_{n_{2, k}})$ так, чтобы вторая координатная последовательность
    сходилась.
    
    Повторим так ещё $d - 2$ раз и получим то, что в векторной 
    подпоследовательности $(x_{n_{k}})$, где $n_k = n_{d, k}$, 
    любая координатная последовательность сходится $\Rightarrow$
    $(x_{n_{k}})$ тоже сходится, т.к. в $\mathbb{R}^d$ покоординатная и
    сходимость по метрике -- одно и то же.
    
    \end{proof}