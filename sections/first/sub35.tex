\subsection{Аналог теоремы Больцано–Вейерштрасса для неограниченной 
    последовательности. Частичные пределы. Теорема о характеристике 
    частичных пределов.}
    
    \begin{theorem-non}\end{theorem-non}
    \begin{enumerate}
        \item Неограниченная монотонная последовательность стремится
        к $+\infty$ или к $-\infty$.
        \item Из любой неограниченной последовательности можно выделить
        подпоследовательность, стремящуюся к $+\infty$ или к $-\infty$.
    \end{enumerate}
    \begin{proof} $ $

    \begin{enumerate}
        \item Пусть $(x_n)$ возрастает. $(x_n)$ неограничена $\Rightarrow$
        никакое $u$ не является верхней границей $\Rightarrow \exists m :
        x_m > u \Rightarrow u < x_m \leqslant x_{m + 1} \leqslant x_{m + 2} \leqslant \dots
        \Rightarrow x_n > u$, начиная с некоторого номера $\Rightarrow
        \lim x_n = +\infty$
    
        \item Пусть $(x_n)$ неограничена сверху.
        
        $1$ не является верхней границей $\Rightarrow \exists x_{n_1} > 1$;\\
        $\max\{2, x_1, x_2, \dots, x_{n_1}\}$ не является верхней границей
        $\Rightarrow \exists x_{n_2} > \max\{\dots\} \Rightarrow x_{n_2} > 2,\\
        n_2 > n_1$;\\
        $\max\{3, x_1, x_2, \dots, x_{n_2}\}$ не является верхней границей
        $\Rightarrow \exists x_{n_3} > \max\{\dots\} \Rightarrow x_{n_3} > 3,\\
        n_3 > n_2$;\\
        и т.д.
    
        Итого, $x_{n_k} > k$ и $n_1 < n_2 < \dots \Rightarrow (x_{n_k})$
        -- подпоследовательность $(x_n)$ и $\lim x_{n_k} = +\infty$ по
        предельному переходу в неравенстве.
    \end{enumerate}
    \end{proof}
    
    \begin{conj}
    $a$ -- частичный предел последовательности $(x_n)$, если найдётся
    подпоследовательность $x_{n_k} \rightarrow a$.
    \end{conj}
    \begin{theorem-non}
    $a$ -- частичный предел последовательности $\Leftrightarrow$
    в любой окрестности точки $a$ найдётся бесконечное число членов
    последовательности.
    \end{theorem-non}
    \begin{proof} $ $
    
        ''$\Longrightarrow$'':
    
        Если $a = \lim x_{n_k}$ и $U_a$ -- окрестность точки $a$, то
        все $x_{n_k}$ кроме конечного числа лежат в $U_a \Rightarrow$
        в $U_a$ лежит бесконечное число членов последовательности $(x_n)$.
    
        ''$\Longleftarrow$'':
    
        Будем строить подпоследовательность, имеющую предел $a$.
    
        В $B_{1}(a)$ найдётся бесконечное число членов последовательности,
        возьмём какой-то и назовём его $x_{n_1}$.\\
        В $B_{1/2}(a)$ найдётся бесконечное число членов
        последовательности, значит найдётся член $(x_n)$ с индексом, большим
        $n_1$, назовём его $x_{n_2}$.\\
        В $B_{1/3}(a)$ найдётся бесконечное число членов
        последовательности, значит найдётся член $(x_n)$ с индексом, большим
        $n_2$, назовём его $x_{n_3}$.\\
        $\dots$
    
        $n_1 < n_2 < n_3 < \dots$\\
        $x_{n_k} \in B_{1/k}(a) \Rightarrow \rho(x_{n_k}, a) < \frac1k
        \Rightarrow \rho(x_{n_k}, a) \rightarrow 0 \Rightarrow
        \lim x_{n_k} = a$
    
    \end{proof}