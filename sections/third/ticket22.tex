\subsection{Длина пути и длина кривой. Определение и простейшие свойства. Аддитивность длины кривой}

\begin{conj}
    Пусть $\gamma\colon [a, b] \to X$. Тогда
    длиной пути называется
    \begin{equation*}
      l(\gamma) = \sup \sum\limits_{k = 1}^{n} \rho(\gamma(t_{k - 1}), \gamma(t_k))
    \end{equation*}
    где $a = t_0 < t_1 < t_2 < \dotsb < t_n = b$.
  
  \end{conj}
  
  \begin{notice}
    Длины эквивалентных путей и длины противоположных путей равны.
  \end{notice}
  
  \begin{conj}
    Длина кривой --- длина пути из класса эквивалентности.
  \end{conj}
  
  \textbf{Свойства}.
  \begin{enumerate}
    \item Длина кривой $\geq \overbrace{\text{длина хорды, соединяющей ее начало и конец}}^{\mathclap{\text{
      расстояние между началом и концом пути
    }}}$.
    \item Длина кривой $\geq$ длина вписанной в нее ломаной.
  \end{enumerate}
  
  \begin{theorem}[аддитивность длины]
    Пусть $\gamma\colon [a, b] \to X, \, c \in [a, b]$ и $\widetilde{\gamma} \coloneqq \gamma\smash{\big |_{[a, c]}}, \; \vardbtilde{\gamma} \coloneqq \gamma \smash{\big|_{[c, b]}}$.
    Тогда
    \begin{equation*}
      l(\gamma) = l(\widetilde{\gamma}) + l(\vardbtilde{\gamma})
    \end{equation*}
  \end{theorem}
  \begin{proof}
    \begin{enumerate}
      \item[]
      \item[,,$\geq$''] Докажем, что $l(\gamma) \geq l(\widetilde{\gamma}) + l(\vardbtilde{\gamma})$. Пусть $a = t_0, t_1, \dotsc, t_n = c = u_0, u_1, \dots, u_m = b$ --- дробление отрезка~$[a, b]$. Тогда
      \begin{equation*}
        \sum\limits_{k = 1}^{n} \rho(\gamma(t_{k - 1}), \gamma(t_{k})) +
        \sum\limits_{k = 1}^{m} \rho(\gamma(u_{k - 1}), \gamma(u_{k})) \leq l(\gamma)
      \end{equation*}
      Заменим каждую сумму на супремум и получим нужно неравенство $ l(\widetilde{\gamma}) + l(\vardbtilde{\gamma}) \leq l(\gamma)$.
      \begin{notice}
        \textit{(от редакторов конспекта)} Мы можем переходить к супремуму просто по определению. Оставим то что мы хотим заменить на супремум в левой части, все остальное перенесем направо. Получим
        \begin{equation*}
        \sum\limits_{k = 1}^{n} \rho(\gamma(t_{k - 1}), \gamma(t_{k})) \leq l(\gamma) -
        \sum\limits_{k = 1}^{m} \rho(\gamma(u_{k - 1}), \gamma(u_{k}))
        \end{equation*}
        Тогда выражение справа это верхняя граница для всех выражений слева такого вида. Поэтому разумеется наименьшая из верхних границ тоже не больше этого выражения. Отсюда получаем
        \begin{equation*}
          \begin{gathered}
              l(\widetilde{\gamma}) = \sup \sum\limits_{k = 1}^{n} \rho(\gamma(t_{k - 1}), \gamma(t_{k})) \leq l(\gamma) -
              \sum\limits_{k = 1}^{m} \rho(\gamma(u_{k - 1}), \gamma(u_{k})) \\
              l(\widetilde{\gamma}) +
              \sum\limits_{k = 1}^{m} \rho(\gamma(u_{k - 1}), \gamma(u_{k})) \leq l(\gamma)
          \end{gathered}
        \end{equation*}
        Аналогично заменяем на супремум второе слагаемое и неравенство доказано.
      \end{notice}
      \item[,,$\leq$''] Докажем, что $l(\gamma) \leq l(\widetilde{\gamma}) + l(\vardbtilde{\gamma})$. Пусть $a = t_0, t_1, \dotsc, t_n = b$ --- дробление отрезка $[a, b]$ и $t_m \leq c < t_{m + 1}$. Тогда
      \begin{equation*}
        \begin{gathered}
          \sum\limits_{k = 1}^{n} \rho(\gamma(t_{k - 1}), \gamma(t_k))
          \leq \\ \leq
          \underbrace{\sum\limits_{k = 1}^{m} \rho(\gamma(t_{k - 1}), \gamma(t_k)) + \rho(\gamma(t_m), \gamma(c))}_{\mathclap{\leq l(\widetilde{\gamma})}} +
          \underbrace{\rho(\gamma(c), \gamma(t_{m + 1})) + \sum\limits_{\mathclap{k = m + 2}}^{n} \rho(\gamma(t_{k - 1}), \gamma(t_k))}_{\mathclap{\leq l(\vardbtilde{\gamma})}} \leq l(\widetilde{\gamma}) + l(\vardbtilde{\gamma})
        \end{gathered}
      \end{equation*}
      Переходим к супремуму и получаем нужно неравенство $l(\gamma) \leq l(\widetilde{\gamma}) + l(\vardbtilde{\gamma})$.
      \end{enumerate}
      Таким образом $l(\gamma) \leq l(\widetilde{\gamma}) + l(\vardbtilde{\gamma})$ и $l(\gamma) \geq l(\widetilde{\gamma}) + l(\vardbtilde{\gamma})$, а значит $l(\gamma) = l(\widetilde{\gamma}) + l(\vardbtilde{\gamma})$. Что и требовалось доказать.
  \end{proof}
  