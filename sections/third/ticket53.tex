\subsection{Равномерная сходимость степенного ряда. Непрерывность суммы степенного ряда. Теорема Абеля}
Оказывается, что внутри круга еще есть равномерная сходимость.

\begin{theorem}
    Пусть $R$ -- радиус сходимости ряда $\sum\limits_{n=0}^\infty a_nz^n$ и $0 < r < R$. 
    Тогда в круге $|z| \leqslant r$ ряд сходится равномерно.
\end{theorem}
\begin{proof}
    $r < R \Rightarrow \sum\limits_{n=0}^\infty a_nr^n$ сходится абсолютно (этот факт был в одном из замечаний).
    Если $|z| \leqslant r$, то $|a_nz^n| \leqslant |a_nr^n|$. 
    Тогда по признаку Вейерштрасса ряды $\sum\limits_{n=0}^\infty a_nz^n$ и $\sum\limits_{n=0}^\infty |a_nz^n|$ сходятся равномерно.
\end{proof}
\begin{follow}
    Сумма степенного ряда непрерывна в круге сходимости.
\end{follow}
\begin{proof}
    Пусть мы хотим доказать непрерывность в точке $z_0$, принадлежащей кругу сходимости.
    Возьмем $r: 0 < |z_0| < r < R$, тогда ряд равномерно сходится в круге $|z| \leqslant r \Rightarrow$ его сумма непрерывна в точках $|z| < r \Rightarrow$ в точке $z_0$ есть непрерывность.
\end{proof}

\begin{theorem} (Абеля) \;
    Пусть $R$ -- радиус сходимости и ряд сходится при $z = R$. 
    Тогда ряд сходится равномерно на $[0, R]$.
\end{theorem}
\begin{proof}
    Рассмотрим ряд $\sum\limits_{n = 0}^\infty a_n x^n$, где $x \in [0, R]$.
    Мы можем его переписать как:
    \begin{gather*}
        \sum\limits_{n = 0}^\infty a_n x^n = \sum\limits_{n = 0}^\infty a_n R^n \left(\frac{x}{R}\right)^n
    \end{gather*}
    Заметим, что $\sum\limits_{n = 0}^\infty a_n R^n$ сходится (по условию), причем равномерно, так как от $x$ не зависит. 
    Также последовательность $(\frac{x}{R})^n$ равномерно ограничена (например, числом 1) и монотонна как геометрическая прогрессия. 
    Тогда по признаку Абеля ряд $\sum\limits_{n = 0}^\infty a_n x^n$ сходится равномерно на $[0, R]$.
\end{proof}

\begin{follow}
    Пусть $R$ -- радиус сходимости и ряд сходится при $z = R$. 
    Тогда его сумма непрерывна на $[0, R]$ и 
    \begin{gather*}
        \lim\limits_{x \to R-} \sum\limits_{n=0}^\infty a_nx^n = \sum\limits_{n=0}^\infty a_nR^n
    \end{gather*}
\end{follow}
\begin{proof}
    По предыдущему следствию знаем, что сумма будет непрерывна. 
    Тогда равенство $\lim\limits_{x \to R-} \sum\limits_{n=0}^\infty a_nx^n = \sum\limits_{n=0}^\infty a_nR^n$ и есть условие непрерывности, так как справа стоит значение в точке $R$.
\end{proof}