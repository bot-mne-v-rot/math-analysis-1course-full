\subsection{Непрерывная дифференцируемость. Определение и равносильное ей свойство}

\begin{conj}
    Пусть $f : E \to \R^m$, $E \subset \R^n$, $a \in \Int E$.
    Тогда $f$ \textbf{непрерывно дифф.} в точке $a$, 
    если $f$ дифф. в окрестности точки 
    $a$ и $d_x f$ непрерывен в точке $a$, т.е. $\norm{d_x f - d_a f}
    \to 0$ при $x \to a$.
\end{conj}

\begin{theorem}
    $f$ непрерывно дифф. $\Longleftrightarrow$ $f$ дифф. в окрестности
    точки $a$ и все её частные производные непрерывны в точке $a$.
\end{theorem}
\begin{proof} $ $

    \begin{itemize}
        \item[``$\Longleftarrow$'':]
        $$ d_x f =
        \begin{pmatrix*}
            \frac{df_1}{dx_1}(x) & \frac{df_1}{dx_2}(x) &
            \dots & \frac{df_1}{dx_n}(x) \\
            \frac{df_2}{dx_1}(x) & \frac{df_2}{dx_2}(x) &
            \dots & \frac{df_2}{dx_n}(x) \\
            \vdots & \vdots & \ddots & \vdots \\
            \frac{df_m}{dx_1}(x) & \frac{df_m}{dx_2}(x) &
            \dots & \frac{df_m}{dx_n}(x) \\
        \end{pmatrix*}$$

        Нужно доказать, что $\norm{d_x f - d_a f} \to 0$ при $x \to a$.
        Оценим квадрат нормы этого линейного оператора:
        $$ \norm{d_x f - d_a f}^2 \leqslant 
        \sum_{j, k} \left( \frac{df_k}{dx_j}(x) -
        \frac{df_k}{dx_j}(a) \right)^2 $$

        $\frac{df_k}{dx_j}(x)$ непрерывна в точке $a$, поэтому:
        $$ \frac{df_k}{dx_j}(x) - \frac{df_k}{dx_j}(a) \to 0
        \text{ при } x \to a $$

        \item[``$\Longrightarrow$'':]
    
        $$ \underbrace{\begin{pmatrix*}
            \frac{df_1}{dx_j}(x) \\
            \frac{df_2}{dx_j}(x) \\
            \vdots \\
            \frac{df_m}{dx_j}(x)
        \end{pmatrix*}}_{\text{$j$-й столбец $d_x f$}} =
        \begin{pmatrix*}
            \frac{df_1}{dx_1}(x) & \frac{df_1}{dx_2}(x) &
            \dots & \frac{df_1}{dx_n}(x) \\
            \frac{df_2}{dx_1}(x) & \frac{df_2}{dx_2}(x) &
            \dots & \frac{df_2}{dx_n}(x) \\
            \vdots & \vdots & \ddots & \vdots \\
            \frac{df_m}{dx_1}(x) & \frac{df_m}{dx_2}(x) &
            \dots & \frac{df_m}{dx_n}(x) \\
        \end{pmatrix*} \cdot
        \begin{pmatrix*}
            0 \\ \vdots \\ 0 \\ 1 \\ 0 \\ \vdots \\ 0
        \end{pmatrix*}
        \begin{matrix*}
            $ $ \\ $ $ \\ $ $ \\
            \longleftarrow \text{$j$-е место}
            \\ $ $ \\ $ $ \\ $ $
        \end{matrix*} = d_x f (e_j)$$

        Таким образом:
        \begin{gather*}
            \abs{\frac{d f_k}{d x_j}(x) - \frac{d f_k}{d x_j}(a)}
            = \norm{\begin{pmatrix*}
                0 \\ \vdots \\ 0 \\ 
                \frac{d f_k}{d x_j}(x) - \frac{d f_k}{d x_j}(a) 
                \\ 0 \\ \vdots \\ 0
            \end{pmatrix*}}
            \leqslant
            \norm{\begin{pmatrix*}
                \frac{df_1}{dx_j}(x) - \frac{df_1}{dx_j}(a) \\
                \frac{df_2}{dx_j}(x) - \frac{df_2}{dx_j}(a) \\
                \vdots \\
                \frac{df_m}{dx_j}(x) - \frac{df_m}{dx_j}(a)
            \end{pmatrix*}} = \\
            = \norm{d_x f(e_j) - d_a f(e_j)} =
            \norm{(d_x f - d_a f)(e_j)} \leqslant
            \norm{d_x f - d_a f} \cdot \underbrace{\norm{e_j}}_{= 1}
            = \norm{d_x f - d_a f} \xrightarrow[x \to a]{} 0
        \end{gather*}
    \end{itemize}
    \textit{Возможно, это доказательство записано излишне подробно.
    Всякие рассуждения с матрицами на экзамене можно проговаривать
    устно.}
\end{proof}

\begin{theorem}
    Непрерывная дифф. сохраняется при взятии лин. комбинации,
    композиции, скал. произведения.
\end{theorem}
\begin{proof}
    Предыдущая теорема говорила о том, что непрерывная 
    дифференцируемость в точке эквивалентна непрерывности всех
    коэффициентов матрицы дифференциала в этой точке.

    Вспомним, что матрицы дифференциалов при взятии лин. комбинации,
    композиции, скал. произведения -- это какие-то линейные комбинации
    или произведения матриц дифференциалов исходных функций. Если
    все коэфф. исходных матриц были непрерывны, то после перемножения
    или сложения все коэфф. итоговой матрицы тоже будут непрерывны.
\end{proof}
