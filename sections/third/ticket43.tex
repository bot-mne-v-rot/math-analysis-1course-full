\subsection{Равномерный предел непрерывных функций. Теорема Стокса–Зайделя. Пространство $C(K)$ и его полнота}
У равномерной сходимости есть много полезных свойств.
Одно их них заключается в том, что при равномерной сходимости сохраняется непрерывность.

\begin{theorem}
    Пусть $f_n: E \to \R$ непрерывны в $(\cdot) \, a \in E$ и $f_n \doublerightarrow f$ на $E$. 
    Тогда $f$ непрерывна в $(\cdot) \, a$.
\end{theorem}
\begin{proof}
    Зафиксируем $\varepsilon > 0$. 
    Распишем равномерную сходимость:
    $\exists N \; \forall n \geqslant N: \; \forall x \in E \;\; \abs{f_n(x) - f(x)} < \varepsilon$.
    Оставим только $N$-тую функцию: $\forall x \in E \;\; \abs{f_N(x) - f(x)} < \varepsilon$.
    Напишем следующее неравенство треугольника $\forall x \in E$:
    \[ \abs{f(x) - f(a)} \leqslant \underbrace{\abs{f(x) - f_N(x)}}_{< \varepsilon} + \abs{f_N(x) - f_N(a)} + \underbrace{\abs{f_N(a) - f(a)}}_{< \varepsilon} < 2\varepsilon + \abs{f_N(x) - f_N(a)} \]
    \quad Для оценки $\abs{f_N(x) - f_N(a)}$ воспользуемся непрерывностью $f_N$ в $(\cdot) \, a$: 
    \[ \exists \delta > 0: \; \forall x \in E: \; \abs{x - a} < \delta \;\;\; \abs{f_N(x) - f_N(a)} < \varepsilon \]
    \quad Таким образом, мы взяли произвольный $\varepsilon$ и смогли подобрать такое $\delta$, что если $\abs{x - a} < \delta$, то $\abs{f(x) - f(a)} < 3\varepsilon$.
    Это и есть критерий непрерывности $f$ в $(\cdot) \, a$.
\end{proof}

\vspace*{4mm}

\follow \; (т. Стокса-Зайделя)

Если $f_n \in C(E)$ -- непрерывны на $E$ и $f_n \doublerightarrow f$ на $E$, то $f \in C(E)$.
Действительно, в каждой точке непрерывность сохраняется, поэтому сохраняется и общая непрерывность.

\vspace*{4mm}

\notice \, Поточечной сходимости не хватает для сохранения непрерывности.

Разберем пример: $f_n(x) = x^n : [0, 1] \to \R, f(x) = \begin{cases} 
    1, & \text{если } x = 1 \\ 
    0, & \text{иначе } x \in [0, 1)
\end{cases}$

$f_n(x) \in C[0, 1]$, а вот предельная функция непрерывной не является.

\vspace*{7mm}

\begin{conj}
    Нормированное пространство непрерывных функций.
    \[ C(K) := \{ f: K \to \R; f - \text{ непрерывна} \}, \text{ где } K - \text{ компакт } \]

    Введем на этом пространстве следующую норму:
    \[ \norm{f}_{C(K)} := \max_{x \in K}{\abs{f(x)}} = \sup_{x \in K}{\abs{f(x)}} \] 

    Заметим, что $C(K)\subset l^{\infty}(K)$, так как функция, непрерывная на компакте, ограничена, и нормы у этих пространств совпадают.
\end{conj}

\vspace*{5mm}

\follow \; (из т. Стокса-Зайделя)

$C(K)$~--- замкнутое подпространство $\ell^{\infty}(K)$.

\begin{proof}
    Берём $f_n \in C(K)$ и $\norm{f_n - f} \to 0$.
    Надо доказать, что $f \in C(K)$.
    Это прямое следствие теоремы Стокса-Зайделя, ведь $\norm{f_n - f} \to 0 \Leftrightarrow f_n \doublerightarrow f$.
\end{proof}

\vspace*{5mm}

Логично предположить, что $C(K)$ аналогично $\ell^{\infty}(K)$ будет полным пространством.
Чтобы доказать это, сформулируем чуть более общую теорему.

\begin{theorem}
    Замкнутное подпространство полного пространства -- полное пространство.
\end{theorem}
\begin{proof}
    Пусть $X$ -- полное пространство, а $Y$ -- замкнутое подпространство в $X$.
    Возьмём фундаментальную последовательность в $Y: a_n \in Y \Longrightarrow a_n$ -- фундаментальна в $X$.
    Воспользовавшись полнотой $X$, заключаем, что $\lim a_n =  a \in X$.
    А так как $Y$ замкнутое, $a$ будет лежать в $Y$, поэтому $Y$ -- полное.
\end{proof}

\follow \; $C(K)$~--- полное нормированное пространство.

