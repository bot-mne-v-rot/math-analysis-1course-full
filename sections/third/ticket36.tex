\subsection{Теорема Коши. Произведение рядов. Теорема Мертенса (без доказательства). Необходимость условия абсолютной сходимости}
\begin{theorem}
    (Теорема Коши (про произведение рядов)) 
    
    Если $\sum a_n = A$ и $\sum b_n = B$ абсолютно сходятся, то
    ряд, составленный из $a_kb_n$ в произвольном порядке будет абсолютно сходиться, и его сумма будет равна $AB$.
\end{theorem}
\begin{proof}
    Пусть $\sum |a_n| = A^*,\ \sum|b_n| = B^*$. Частичная сумма $\sum |a_i b_j| \leq A^*_{\max\{i\}}B^*_{\max\{j\}}$. Проверим данное утверждение:
    \begin{gather*}
        (|a_1| + |a_2| + \ldots + |a_n|)(|b_1| + |b_2|+\ldots + |b_m|) = \sum \limits_{i=1}^n \sum \limits_{j=1}^m |a_i b_j| \geq \sum|a_i b_j|
    \end{gather*}
    При этом $A^*_{\max\{i\}} B^*_{\max\{j\}} \leqslant A^* B^*$. То есть все частичные суммы 
    меньше либо равны $A^*B^*$, значит ряд абсолютно сходится (так как рассматриваем неотрицательные слагамые).

    Ряд абсолютно сходится, значит можно переставлять слагаемые в произвольном порядке.
    Мы можем смотреть на подпоследовательность частичных сумм, то есть группировать
    члены ряда и смотреть на сгруппированные части. Тогда сгруппируем члены ряда так, 
    чтобы получить нужную нам сумму.

    Будем выбирать частичные суммы так: запишем все $a_ib_j$ в табличку,
    с соответствующей нумерацией столбцов и строк. На $k$-м шаге
    будем в частичную сумму брать $k^2$ слагаемых из верхнего
    левого квадрата $k\times k$ таблицы. То есть:
      
    $S_1$ На первом шаге:
    \begin{center}
        $\begin{matrix}
            a_1b_1
        \end{matrix}$
    \end{center}
    $S_2$ На втором шаге плюс 3 слагамых:
    \begin{center}
        $\begin{matrix}
            a_1b_1 & a_1b_2 \\
            a_2b_1 & a_2b_2
        \end{matrix}$
    \end{center}
    $S_3$ На третьем шаге плюс 5 слагаемых:
    \begin{center}
        $\begin{matrix}
            a_1b_1 & a_1b_2 & a_1b_3 \\
            a_2b_1 & a_2b_2 & a_2b_3 \\
            a_3b_1 & a_3b_2 & a_3b_3
        \end{matrix}$
    \end{center}
    Формально:
    \begin{equation*}
        S_n = \sum_{i,j \leq n} a_ib_j = \sum_{i=1}^n a_i \sum_{j=1}^n b_j
        = A_nB_n \longrightarrow AB
    \end{equation*}

\end{proof}

\begin{conj}
    Произведением рядов $\sumn a_n, \sumn b_n$ называется $\sumn c_n$, где 
    \begin{gather*} 
        c_n = a_1b_n + a_2b_{n-1}+a_3b_{n-2} + \ldots + a_nb_1
    \end{gather*}
\end{conj}

\textbf{Мотивация группировать именно так}: рассмотрим степенные ряды $\sumn a_nt^n$ и $\sumn b_nt^n$
\begin{gather*}
    a_kt^k b_nt^n = a_kb_n t^{k+n}
\end{gather*}
Тогда логично сгруппировать всё с одинаковой степенью $t$. Тогда получаем:
\begin{gather*}
   \sum\limits_{k+n = m}a_kb_n t^m
\end{gather*}
А это и есть произведение рядов, потому что сумма индексов у $a$ и $b$ фиксированная. 

\begin{theorem}
    Теорема Мертенса (без доказательства). 
    
    Если $\sum a_n = A$ и $\sum b_n = B$ сходятся, 
    причем один из них абсолютно сходится, то $\sum c_n$ сходится и $\sum c_n = AB$
\end{theorem}

\underline{\textit{Замечания:}}
\begin{enumerate}
    \item[1.] Здесь важен порядок суммирования
    \item[2.] Просто сходимости не хватает 
\end{enumerate}

\textbf{Пример}
\begin{equation*}
    \sumn \frac{(-1)^{n-1}}{\sqrt{n}}
\end{equation*}
Знаем, что сходится по признаку Лейбница.

Умножим его на себя. 
\begin{align*}
    c_n &= a_1b_n + a_2b_{n-1} + \ldots + a_nb_1 \\
    &= \frac{(-1)^0}{\sqrt{1}} \frac{(-1)^{n-1}}{\sqrt{n}} + 
    \frac{(-1)^1}{\sqrt{2}}\frac{(-1)^{n-2}}{\sqrt{n-1}} + \ldots +
    \frac{(-1)^{n-1}}{\sqrt{n}}\frac{(-1)^0}{\sqrt{1}} \\
    &= (-1)^{n-1} (\frac{1}{\sqrt{1}\sqrt{n}} + 
    \frac{1}{\sqrt{2}\sqrt{n-1}} + \ldots + \frac{1}{\sqrt{n}\sqrt{1}})
\end{align*}
Поймем, что $|c_n|$ будет большим. $\sqrt{k}\sqrt{n+1-k} = \sqrt{k(n+1)-k^2} \leq \frac{n+1}{2}$, 
так как произведение максимально при фиксированной сумме, когда сомножители равны.
Тогда:
\begin{gather*}
    \frac{1}{\sqrt{k}\sqrt{n+1-k}} \geq \frac{2}{n+1} \\
    \Longrightarrow |c_n| \geq n \cdot \frac{2}{n+1} \geq 1 \\
    \Longrightarrow \text{ ряд расходится}
\end{gather*}