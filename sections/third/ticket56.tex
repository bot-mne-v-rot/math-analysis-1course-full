\subsection{Формула для коэффициентов разложения в ряд аналитической функции. Несовпадение классов бесконечно дифференцируемых и аналитических функций}
\begin{theorem}
    (Единственность разложения функции в степенной ряд)
    
    Пусть есть функция, разложенная в степенной ряд:
    \begin{gather*}
        f(z) = \sumi a_n (z - z_0)^n \text{ при } \abs{z - z_0} < R
    \end{gather*}
    Тогда:
    \begin{gather*}
        a_n = \frac{f^{(n)}(z_0)}{n!}
    \end{gather*}
\end{theorem}
\begin{proof}
    По только что доказанной теореме: 
    \begin{gather*}
        f^{(m)}(z) = \sum\limits_{n=m}^\infty n(n-1)\dots (n-m+1)a_n (z-z_0)^{n-m}
    \end{gather*}
    Если посмотреть на $f^{(m)}(z_0)$, то мы видим, что все слагаемые зануляются из-за последнего множителя. За исключением случая $n=m$. Тогда мы получаем:
    \begin{gather*}
        f^{(m)}(z_0) = m(m-1) \dots 1 \cdot a_m = a_m \cdot m!
    \end{gather*}
    Получили нужную формулу для коэффициентов. Единственность следует из единственности производной.
\end{proof}
\begin{conj}
    Ряд Тейлора для функции $f$ в точке $z_0$:
    \begin{gather*}
        \sumi \frac{f^{(n)}(z_0)}{n!} (z - z_0)^n
    \end{gather*}
\end{conj}
\begin{conj}
    Функция называется \textbf{аналитической} в точке $z_0$, если она бесконечно 
    дифференцируема и является суммой своего ряда Тейлора в некой окрестности $z_0$:
    \begin{gather*}
       f(z) = \sumi \frac{f^{(n)}(z_0)}{n!} (z - z_0)^n 
    \end{gather*}
\end{conj}
\notice \; Просто бесконечной дифференцируемости функции в общем случае недостаточно для аналитичности.

\textbf{Пример:} 
\begin{gather*}
    f(x) = \begin{cases}
        e^{-1/x^2} \text{ при } x \neq 0 \\
        0 \text{ при } x = 0
    \end{cases} 
\end{gather*} 
Утверждаем, что она бесконечно дифференцируема, но не раскладывается в свой ряд Тейлора в нуле. 
\begin{proof}
    Для начала посмотрим на производные в точках, отличных от нуля.
    По индукции докажем, что при $x \neq 0$:
    \begin{gather*}
        f^{(n)}(x) = \frac{P_n(x)}{x^{3n}} \cdot e^{-1/x^2}
    \end{gather*}
    Где $P_n(x)$ -- некий многочлен. База очевидна, дробь слева равна 1. Переход $n \longrightarrow n + 1$:
    \begin{gather*}
        f^{(n+1)}(x) = \left( \frac{P_{n}(x)}{x^{3n}} e^{-1/x^2} \right)' = \left( P_{n}(x)x^{-3n} e^{-1/x^2} \right)' = \\
        P_{n}'(x)x^{-3n} e^{-1/x^2} + P_{n}(x)(-3n x^{-3n-1}) e^{-1/x^2} + P_{n}(x)x^{-3n} e^{-1/x^2} \cdot \frac{2}{x^3} = \\ 
        \frac{e^{-1/x^2}}{x^{3n+3}}(x^3 P_n'(x) - 3nx \cdot P_n(x) + P_n(x))
    \end{gather*}
    Теперь дифференцируем в нуле. Докажем по индукции, что $f^{(n)}(0) = 0$. База очевидна из условия. Переход $n \longrightarrow n+1$:
    \begin{gather*}
        f^{(n+1)}(0) = \lim\limits_{x \rightarrow 0} \frac{f^{(n)}(x) - \stackabove{f^{(n)}(0)}{0}}{x} = \lim\limits_{x \rightarrow 0} \frac{P_n(x)e^{-1/x^2}}{x^{3n+1}} = 
        \lim\limits_{x \rightarrow 0} P_n(x) \cdot \underbrace{\colorboxed{red}{\frac{e^{-1/x^2}}{x^{3n+1}}}}_{\rightarrow 0} = 0
    \end{gather*}
    Что мы получили? Мы получили дифференцируемость во всех точках.
    Теперь попробуем разложить $f$ в ряд Тейлора в нуле. Это какая-то хрень из нулей потому что в нуле все производные ноль:
    \begin{gather*}
        \sumi 0 \cdot x^n \equiv 0 
    \end{gather*}
    Это очевидно неверно, значит $f$ не является суммой своего ряда Тейлора, а значит она не аналитична. 
\end{proof}