\subsection{Формула Стирлинга}
\begin{enumerate}
    \setcounter{enumi}{2}
    \item Формула Стирлинга.  Распишем $\ln n! = \ln 1 + \ln 2 + \dotsb + \ln n$ по формуле Эйлера-Маклорена при $f(x) = \ln x, \, f'(x) = \frac{1}{x}, \, f''(x) = -\frac{1}{x^2}, \, m = 1$.
    \begin{equation*}
        \ln n! = \underbracket{\frac{\ln 1 + \ln n}{2}}_{\mathclap{\frac{\ln n}{2}}} +
        \int_{1}^{n} \ln t \: dt -
        \frac{1}{2}\underbrace{\int_{1}^{n} \frac{\{t\}(1 - \{t\})}{t^2}}_{\mathclap{=: b_n}} \: dt
    \end{equation*}
    Заметим, что
    \begin{equation*}
        \int_{1}^{n} \ln t \: dt = t\ln t \Big|_{1}^{n} - \int_{1}^{n} \frac{1}{t} \cdot t \: dt = n \ln n - (n - 1)
    \end{equation*}
    Подставим полученное выражение обратно в формулу:
    \begin{equation*}
        \ln n! = \frac{\ln n}{2} + n\ln n - n + 1 - \frac{b_n}{2}
    \end{equation*}
    Теперь воспользуемся тем же приемом, что и в прошлом примере. Заметим, что $b_n$ растут, оценим их сверху:
    \begin{equation*}
        b_n = \int_{1}^{n} = \frac{\{t\}(1 - \{t\})}{t^2} \: dt \leq
        \frac{1}{4} \int_{1}^{n} \frac{dt}{t^2} = \frac{1}{4}\left(-\frac{1}{t}\right)\Big|_{1}^{n} = \frac{1}{4}\left(1 - \frac{1}{n}\right) < \frac{1}{4}
    \end{equation*}
    Поэтому $b_n = b + o(1)$. Подставим полученное тождество в формулу для $\ln n!$
    \begin{equation*}
        \ln n! = n\ln n + \frac{\ln n}{2} - n + C + o(1)
    \end{equation*}
    Возьмем экспоненты от обеих частей равенства:
    \begin{equation*}
        n! = n^{n}\sqrt{n}e^{-n}e^{C}e^{o(1)}
    \end{equation*}
    Расписав $e^{o(1)} = 1 + o(1)$ по формуле Тейлора получаем, что
    \begin{equation*}
        n! \sim e^Cn^ne^{-n}\sqrt{n}
    \end{equation*}
    Попробуем вычислить константу $C$. Для этого воспользуемся приближением для $C_{2n}^{n}$:
    \begin{equation*}
        C_{2n}^{n} \sim \frac{4^n}{\sqrt{\pi n}}
    \end{equation*}
    С другой стороны мы можем расписать $C_{2n}^{n}$ по определению и каждый из факториалов заменить на полученное нами ранее выражение.
    \begin{equation*}
        C_{2n}^{n} = \frac{(2n)!}{(n!)^2} \sim \frac{e^C(2n)^{2n}e^{-2n}\sqrt{2n}}{(e^Cn^ne^{-n}\sqrt{n})^2} =
        \frac{\cancel{e^C}2^{2n}\cancel{n^{2n}}\cancel{e^{-2n}}\sqrt{2\cancel{n}}}{e^{\cancel{2}C}\cancel{n^{2n}}\cancel{e^{-2n}}\sqrt{n}\sqrt{\cancel{n}}} =
        \frac{4^n\sqrt{2}}{e^C\sqrt{n}}
    \end{equation*}
    Значит
    \begin{equation*}
        \frac{4^n}{\sqrt{\pi n}} \sim \frac{4^n \sqrt{2}}{e^C \sqrt{n}}
    \end{equation*}
    Значит предел отношения этих величин равен единице, то есть:
    \begin{equation*}
        \lim\limits_{n \to +\infty} \frac{4^n e^C\sqrt{n}}{4^n \sqrt{\pi n} \sqrt{2}} = 1
        \implies
        \lim\limits_{n \to +\infty} \frac{e^C}{\sqrt{2\pi}} = 1
    \end{equation*}
    Но предел константы это всегда сама константа, значит
    \begin{equation*}
        \frac{e^C}{\sqrt{2\pi}} = 1 \implies e^C = \sqrt{2\pi}
    \end{equation*}
    Подставив $e^C$ получаем формулу Стирлинга:
    \begin{equation*}
        \boxed{n! \sim n^ne^{-n}\sqrt{2\pi n}}
    \end{equation*}
    \begin{notice}
        Можно доказать, что
        \begin{equation*}
            n^ne^{-n}\sqrt{2\pi n} \: e^{\frac{1}{12n + 1}} < n! <
            n^ne^{-n}\sqrt{2\pi n} \: e^{\frac{1}{12n}}
        \end{equation*}
    \end{notice}
\end{enumerate}