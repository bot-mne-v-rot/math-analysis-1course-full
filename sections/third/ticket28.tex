\subsection{Признак Коши (с $\overline{\lim}$). Примеры}
\textbf{Признак Коши.} 
Пусть $a_n \geqslant 0$. Тогда:
\begin{enumerate}
    \item Если $\sqrt[n]{a_n} \leqslant q < 1$, то ряд сходится.
    \item Если $\sqrt[n]{a_n} \geqslant 1$, то ряд расходится.
    \item Пусть $q^* := \overline{\lim}_{n \to \infty} \sqrt[n]{a_n}$.
    Если $q^* < 1$, то ряд сходится, если $q^* > 1$, то ряд расходится, иначе ничего утверждать нельзя.
\end{enumerate}
\begin{proof} \quad

    \begin{enumerate}
        \item $a_n \leqslant q^n$ при $q < 1 \Rightarrow$ применяем признак сравнения с бесконечно убывающей геометрической прогрессией.
        \item $a_n \geqslant 1$, нет необходимого условия сходимости.
        \item Пусть $q^* > 1$. 
        Вспомним, что это наибольший частичный предел и возьмем эту подпоследовательность: $\sqrt[n_k]{a_{n_k}} \to q* > 1$.
        Тогда $\sqrt[n_k]{a_{n_k}} > 1$ при больших $k$ $\Rightarrow a_{n_k} > 1$ при больших $k$.
        Опять не выполняется необходимое условие сходимости.

        Пусть $q^* < 1$. 
        Распишем верхний предел по определению: $q^* = \lim\limits_{n \to \infty} \sup\limits_{k \geqslant n} \sqrt[k]{a_k} < 1$. 
        Значит, с какого-то момента $\sup\limits_{k \geqslant n} \sqrt[k]{a_k} < \frac{q^* + 1}{2}$ (это середина отрезка от $q^*$ до 1).
        Супремум не меньше членов подпоследовательности, поэтому $\sqrt[n]{a_n} < \frac{q^* + 1}{2} < 1$ с какого-то момента.
        Осталось применить первый пункт.
    \end{enumerate}
\end{proof}

\underline{Примеры, когда $q^* = 1:$}
\begin{enumerate}
    \item У гармонического ряда $q^* = 1$, а он расходится.
    \item У ряда $\sum\limits_{n=1}^{\infty} \frac{1}{n(n+1)}$ $q^*$ тоже равен 1, но он сходится. 
\end{enumerate}
