\subsection{Модуль непрерывности. Свойства}

\begin{conj}
    Пусть $f\colon E \to \R$. Тогда модулем непрерывности $\omega_f$ от $\delta \geq 0$ называют
    \begin{equation*}
      \omega_f(\delta) \coloneqq
      \sup\{|f(x) - f(y)| : x, y \in E
      \text{ и }\rho(x, y) \leq
      \delta\}
    \end{equation*}
\end{conj}

\textbf{Свойства.}
\begin{enumerate}
  \item $\omega_f(\delta) \geq 0$ и $\omega_f(0) = 0$
  \item $\omega_f$ нестрого возрастает
  \item $|f(x) - f(y)| \leq \omega_f(\rho(x, y))$
  \item Если $f$ --- липшицева с константой $M$, то $\omega_f(\delta) \leq M\delta$
  \item $f$ равномерно непрерывна на $E \iff \omega_f$ непрерывна в нуле.
  \begin{proof}
    \begin{enumerate}
      \item[]
      \item[] $\boxed{\Rightarrow}$
      Из равномерной непрерывности:
      \begin{equation*}
        \forall \varepsilon > 0 \; \exists \gamma > 0
        \colon \forall x, y \in E\text{ и }\rho(x, y) < \gamma \implies |f(x) - f(y)| < \varepsilon
      \end{equation*}

      Тогда если взять $\delta < \gamma$ (можем так сделать, потому что на интересуют только маленькие $\delta$), то
      \begin{equation*}
        \rho(x, y) \leq \delta < \gamma \implies
        |f(x) - f(y)| < \varepsilon \implies \omega_f(\delta) \leq \varepsilon
        \implies \lim\limits_{\delta \to 0+} \omega_f(\delta) = 0 = \omega_f(0)
      \end{equation*}

      \item[] $\boxed{\Leftarrow}$
      Пусть $\lim\limits_{\delta \to 0+} \omega_f(\delta) = 0$. Тогда
      \begin{equation*}
        \hphantom{\text{, если $\rho(x, y) \leq \delta$}}
        \forall \varepsilon > 0 \; \exists \delta > 0 \colon
        \varepsilon > \omega_f(\delta) \geq |f(x) - f(y)|
        \text{, если $\rho(x, y) \leq \delta$}
      \end{equation*}
      То есть мы поняли, что для любого $\varepsilon$ найдется такое $\delta$, что как только аргументы отличаются не больше чем на $\delta$, значения отличаются меньше чем на $\varepsilon$. Это равномерная непрерывность по определению.
    \end{enumerate}
  \end{proof}
  \item Пусть $K$ --- компакт. Тогда $f \in C(K) \iff \lim\limits_{\delta \to 0+} \omega_f(\delta) = 0$.

  В частности $f \in C[a, b] \iff \lim\limits_{\delta \to 0+} \omega_f(\delta) = 0$.
\end{enumerate}